\myChapter{Usage Control in FACPL}
\label{chap:Usage Control in FACPL}
In questo capitolo si descrive un'estensione del linguaggio \ac{FACPL} basata sul modello \ac{UCON}.
Più precisamente lo sviluppo è stato diviso in due parti. In primis, la struttura del linguaggio è stata modificata di modo
che si possa creare delle politiche che si basino sul comportamento passato. In seguito il linguaggio
è stato ulteriormente esteso per migliorare la gestione dell'utilizzo continuativo di risorse.\par
In questo capitolo e nel succesivo si espone il secondo sviluppo, per la prima parte dell'estensione si rimanda
alla tesi del mio collega Federico Schipani, con cui ho lavorato per l'implementazione dell'intera nuova struttura.
La sua tesi descrive l'implementazione della componente status per tenere traccia del comportamento passato e
le strutture create per utilizzarla nel processo valutativo. \par
Il linguaggio è stato esteso in modo tale da ottimizzare la gestione di un insieme di richieste sulla stessa risorsa.
Per far questo è stata necessaria una modifica sul processo di valutazione~(Sezione \ref{sec:Il processo di valutazione Facpl new})
e l'inserimento di nuovi componenti~(Sezione~\ref{sec:SintassiNew}).
\section{Il processo di valutazione}
\label{sec:Il processo di valutazione Facpl new}
Per memorizzare il comportamento passato, è stato essenziale l'aggiunta di un altro elemento
nel processo di valutazione di Figura \ref{fig:img/evaluationProcess.png}.
Come si vede in Figura \ref{fig:img/evaluationProcessStatus}, la nuova componente è Status.\par
\MyFigure{img/evaluationProcessStatus}{Processo di valutazione Status}{1}
Oltre agli \emph{attribute name} il Context Handler ha anche la possibilità di ricavare gli \emph{status attribute}.
Il \ac{PDP} quindi potrà richiedere entrambi i tipi di attributi per elaborare la sua risposta. La decisione poi passa al PEP
come avveniva precedentemente.\par
Inoltre, sono state aggiunte un tipo di obbligazioni che possono modificare lo Status. Il \ac{PEP} quindi può far eseguire questo tipo di
azioni per cambiare gli attributi.\par
Riprendendo il primo esempio di Sezione~\ref{subs:Accessi in lettura e scrittura di file}, si riporta il seguente requisito
\begin{quote}
\emph{Se uno degli amministratori sta scrivendo su un file, nessuno può leggere o apportare modifiche nello stesso momento}.
\end{quote}
Questa regola è realizzabile
solo con l'utilizzo di uno status attribute (e.g.\ un booleano isWriting) che viene modificato da un'azione del PEP dopo che
una richiesta di scrittura è stata accettata. Il \ac{PDP} nelle successive richieste di lettura o di scritturà farà un controllo
sull'attributo e negherà di conseguenza l'accesso.\par
\MyFigure{img/evaluationProcessPEP}{Processo di valutazione PEP-Check}{1}
Il passo successivo consiste nell'eliminazione della ridondanza dei controlli da parte del \ac{PDP}.\par
Quando in Sezione~\ref{subs:Accessi in lettura e scrittura di file}, si scrive:
\begin{quote}
  \emph{Letture ripetute, su uno stesso insieme di file, vengono gestite in modo efficiente}
\end{quote}
si fa riferimento al nuovo metodo per il controllo di richieste uniformi su una categoria di risorse e
in Figura \ref{fig:img/evaluationProcessPEP} si mostra il processo di valutazione ridotto.\par
Si può definire un certo tipo di richieste che verrà interamente gestito dal \ac{PEP} dopo una prima, e unica,
valutazione del \ac{PDP}. Se un sistema deve gestire numerose richieste di lettura, e non ci sono elementi che bloccano
un accesso, è possibile accettare una richiesta esaminando soltanto la tipologia dell'azione e l'appartenenza di una risorsa ad
un insieme predefinito.\par
Prendiamo per esempio un sistema che gestisce dei file dentro delle cartelle. Supponiamo che ci sia una cartella
"Share" dove tutti gli utenti possono leggere solo se non ci sono utenti che scrivono.
Il sistema può gestire richieste di lettura ripetute, facendo un unico controllo per la verifica dell'assenza di
utenti che stanno scrivendo solo alla prima domanda di accesso. Poi le richieste successive potranno essere
gestite dal \ac{PEP} con soltanto due controlli, uno sul tipo di azione, in questo caso una lettura, l'altro sull'appartenza
alla cartella "Share". Se i due controlli non sono validi si ritorna alla valutazione completa con il \ac{PDP}.\par
Si fa notare che il processo di valutazione ridotto di Figura \ref{fig:img/evaluationProcessPEP} non sostituisce il
processo mostrato in Figura~\ref{fig:img/evaluationProcessStatus},
ma il suo utilizzo porta notevoli vantaggi prestazionali e elimina i controlli ridondanti dalle verifiche
per le autorizzazioni delle richieste.
\section{Sintassi}
\label{sec:SintassiNew}
La modifica nella struttura ha reso necessario apportare dei cambiamenti anche nella grammatica.
La nuova sintassi è riportata in Tabella \ref{tab:facpl_new_syntax}.\par
Al \ac{PAS} è stato aggiunto \status della forma:
$$(status: Attribute^+)^?$$
quindi lo \status è un elemento opzionale e può essere formato da almeno un \emph{ Attribute}.

\emph{ Attribute} invece è della forma:
$$(Type\ Identifier (= Value)^?)$$
quindi è formato da un tipo, che può essere \emph{int}, \emph{float}, \emph{boolean} o \emph{date}, da una stringa
identificatrice e da un valore.\par
Le \emph{PepAction} sono state modificate in modo tale da eseguire operazioni sugli attributi dello stato
e sono della forma: \emph{nomeAzione}(\emph{Attribute},type)
per esempio l'azione di somma di interi è \emph{add}(\emph{Attribute},int).\par
Agli \emph{Attribute Name} è stata aggiunta la produzione \textit{Status/Identifier} per identificare i
termini dello Status nelle \emph{Expression}.\par
La nuova produzione per le \emph{Obligation} della forma:
$$[Effect\ env:Expr\  status:Expr \ expiration:Value^?]$$
è utilizzata per la nuova gestione delle richieste da parte del \ac{PEP} ed è denominata \emph{Obligation Check}.
La nuova tipologia non ha pepAction, in quanto il \ac{PEP} può modificare valori dello \status solo se il
processo di valutazione è quello completo, cioè quello che include anche i controlli eseguiti dal \ac{PDP}.
Inoltre deve esserci un insieme di \emph{Expression} affinché il \ac{PEP} possa fare le proprie verifiche
e può esserci un valore opzionale che indica il tipo della scadenza.\par
Infine \emph{expiration:$Value^?$} della forma:
$$Value ::= Int\ |\ Date$$
indicherà la scadenza delle \emph{Obligation Check}. Un intero (Int) determina il numero massimo di richieste, mentre
una data (Date) il limite temporale.
Se \emph{expiration} viene omesso la \eobligation non ha scadenza.
\begin{table}
\footnotesize

\caption{Sintassi ausiliaria per le risposte}
\hrule
$
\begin{array}{@{\ }r@{\ \ \ \ }r@{\ }r@{\ \ }l@{\ }}

&&&\\[-.2cm]
{\textbf{PDP \ Responses }} &
\mathit{PDPResponse} & ::= & \langle \,\mathit{Decision} \ \ \ \mathit{FObligation}^* \rangle
\\[.2cm]
{\textbf{Decisions}} &
\mathit{Decision} & ::= & \permit \Sep \deny \Sep \notApp \Sep \indet
\\[.2cm]
{\textbf{Fulfilled obligations}} &
\mathit{FObligation} & ::= &  [ \, \mathit{ObType} \ \ \obl{\mathit{Value}} \, ]\\[.1cm]
&& & \Sep [ \, \x{env:} \, \mathit{Expr}\ \  \x{status:}\mathit{Expr}\, ]

\end{array}
$\\
\hrule
\label{tab:facpl_new_context_syntax}
\end{table}
{\label{Sintassi ausiliaria di FACPL Aggiornata}}
\begin{table}
\centering
\scriptsize
\caption{Sintassi di $FACPL_{PB}$}
\hrule
$
\begin{array}{@{\,}r@{\ \ }r@{\ }r@{\ \ }l@{\ }}

&&&\\[-.2cm]
{\textbf{Policy Authorisation Systems}} &
\mathit{PAS} & ::= & ( \,  \x{pep:} \, \mathit{EnfAlg}\ \ \x{pdp:}\, \mathit{PDP} \, \  \x{status:}\, \mathit{
[Attribute]})
\\[.2cm]
{\textbf{Attribute}} &
\mathit{Attribute}
& ::= & (\mathit{Type} \ \mathit{Identifier} \ (= Value )^{?})
\\[.2cm]
{\textbf{Type}} &
\mathit{Type}
& ::= & \x{int} \ | \ \x{boolean} \ | \ \x{date} \ | \ \x{double}
\\[.2cm]
{\textbf{Enforcement algorithms}} &
\mathit{EnfAlg}
& ::= & \based \Sep \denyBiased \Sep \permitBiased
\\[.2cm]
{\textbf{Policy Decision Points}} &
\mathit{PDP} & ::= & \pdpPol{\algNT\ }{\x{policies:} \, \mathit{Policy}^{+}}
\\[.2cm]
{\textbf{Combining algorithms}} &
\algNT & ::= & \permitOver \Sep \denyOver \Sep \denyUnless \Sep \permitUnless \\
&& \mid &
\firstApp \Sep \onlyOneApp \Sep \weakCon \Sep \strongCon
\\[.2cm]
{\textbf{Policies}} &
\mathit{Policy} & ::= &
\ruleOpt{\mathit{Effect}\ \ \x{target:} \, Expr\ \ \x{obl:} \, \mathit{Obligation}^{*} \, } \\
&& \mid &
\{ \algNT\ \ \x{target:} \, Expr\ \
\x{policies:} \, \mathit{Policy}^{+}  \ \ \x{obl:} \, \mathit{Obligation}^{*} \, \}
\\[.2cm]
{\textbf{Effects}} &
\mathit{Effect} & ::= & \permit \Sep \deny
\\[.2cm]
{\textbf{Obligations}} &
\mathit{Obligation} & ::= & [ \, \mathit{Effect} \ \ \mathit{ObType} \ \ \obl{Expr} \, ]\\
&& & \Sep [ \, \mathit{Effect} \ \ \x{env:} \, \mathit{Expr}\ \  \x{status:}\mathit{Expr} \ \ \x{exp:}\mathit{Value}^?\, ]
\\[.2cm]
{\textbf{PepAction}} & \mathit{PepAction} & ::= & \, \x{add(\mathit{Attribute}, int)} \Sep \x{flag(\mathit{Attribute}, boolean)} \\
&& & \Sep \x{sumDate(\mathit{Attribute}, date)} \Sep \x{div(\mathit{Attribute}, int)} \\
&& & \Sep \x{add(\mathit{Attribute}, float)} \ \Sep \x{mul(\mathit{Attribute}, float)} \\
&& & \Sep \x{mul(\mathit{Attribute}, int)} \ \Sep \x{div(\mathit{Attribute}, float)} \\
&& & \Sep \x{sub(\mathit{Attribute}, int)} \ \Sep \x{sub(\mathit{Attribute}, float)} \\
&& & \Sep \x{sumString(\mathit{Attribute}, string)} \\
&& & \Sep \x{setValue(\mathit{Attribute}, string)}\\
&& & \Sep \x{setDate(\mathit{Attribute}, date)}
\\[.2cm]
{\textbf{Obligation Types}} &
\mathit{ObType} & ::= & M \Sep O
\\[.4cm]
\textbf{Expressions}&
\mathit{Expr} & ::= &
\mathit{Name} \Sep \mathit{Value}  \\
& & \mid &\x{and(\mathit{Expr}, \mathit{Expr})} \Sep \x{or(\mathit{Expr}, \mathit{Expr})} \Sep \x{not(\mathit{Expr})} \\
& & \mid &
 \x{equal(\mathit{Expr},\mathit{Expr})}  \Sep \x{in}(\mathit{Expr}, \mathit{Expr}) \\
& & \mid & \x{greater}\textrm{-}\x{than(\mathit{Expr},\mathit{Expr})} \Sep \x{add(\mathit{Expr} ,\mathit{Expr} )}\\
& & \mid & \x{subtract(\mathit{Expr} ,\mathit{Expr} )} \Sep \x{divide(\mathit{Expr} ,\mathit{Expr} )}\\
& & \mid & \x{multiply(\mathit{Expr} ,\mathit{Expr} )}  \Sep \x{less}\textrm{-}\x{than(\mathit{Expr}, \mathit{Expr})}\\
\\[.2cm]
%
\textbf{Attribute Names} &
\mathit{Name} & ::= & \mathit{Identifier}/\mathit{Identifier} \ | \ \x{status}/\mathit{Identifier}\\[.2cm]
%
\textbf{Literal Values} &
\mathit{Value} & ::= & \x{true} \mid \x{false} \mid \mathit{Double} \mid \mathit{String} \mid \mathit{Date}
\\[.4cm]
{\textbf{Requests}} &
\mathit{Request} & ::= & {\attribute{\mathit{Name}}{\mathit{Value}}}^{+}
\\[.1cm]
\end{array}
$
\hrule
\label{tab:facpl_new_syntax}
\end{table}
{\label{Sintassi di FACPL Aggiornata}}

La sintassi delle risposte è rimasta quasi interamente invariata. Nella Tabella \ref{tab:facpl_new_context_syntax}
si può vedere che le decisioni non sono cambiate, mentre c'è una nuova produzione per le \emph{Fulfilled Obligation}.

La nuova forma indica le \emph{Fulfilled Obligation} di tipo Check. Queste hanno solo due \emph{Expression}, una per
l'\emph{Environment} e una per lo \emph{Status}. I due termini esprimono i controlli che deve effettuare il \ac{PEP}
prendendo i valori dal \emph{Context Handler}. Le nuove \emph{Fulfilled Obligation} sono prive di pepAction in quanto,
come enunciato prima, il \ac{PEP} non può eseguire azioni, ma solo verifiche sugli attributi.\par
\vspace{3mm}
Si propone adesso un esempio di una Policy che utilizza il nuovo tipo di obbligazione.
Nella Policy si vuole specificare che un utente di nome \emph{Charlie} può effettuare una lettura
solo se nessuno sta scrivendo (come specificato in linea 5 nel Codice \ref{lst:esempio_sintassiPol})
e vuole utilizzare la risorsa \emph{contabilita.xlsx} (linea 2).
Se la richiesta viene accettata si deve cambiare l'attributo \emph{isReading} a \texttt{true} (linea 7), oltre a ciò al susseguirsi
di un altra richiesta di lettura il sistema cambiarà il tipo di processo valutativo e il \ac{PEP} effettuerà i controlli
solo sul tipo di azione e sul nome della risorsa.\par
\lstinputlisting[language = FACPL,stepnumber=2, caption = {Esempio per la sintassi}\label{lst:esempio_sintassiPol}]{code/cap4/charlieRead}
\clearpage
\lstinputlisting[language = FACPL,stepnumber=2, caption = {Esempio PAS}\label{lst:esempio_sintassiPAS}]{code/cap4/charliePAS}
\vspace{1em}
Alla linea 7 di Codice~\ref{lst:esempio_sintassiPAS} si inizializzano i valori dello status
\emph{isWriting} e \emph{isReading} a \texttt{false}.
Si fa notare inoltre che con questa porzione di codice:
\lstinputlisting[firstline = 8, lastline = 8,language = FACPL]{code/cap4/charlieRead}\label{lst:esempio_sintassi}
\vspace{1em}
si esprime la nuova struttura per la creazione delle \emph{Obligation Check} ed è questo il costrutto
in cui si indicano i controlli che il \ac{PEP} deve effettuare. Come si vede, a differenza delle altre
obbligations, non è specificato il tipo M o O (mandatory o optional) in quanto tutte le Obligation check
devono essere eseguite in ogni caso.
\section{Semantica}
\label{sub:SemanticaNew}
La modifica della sintassi ha determinato anche variazioni nella semantica descritta in Sezione \ref{sec:sem_fpl}.\par
Il \ac{PDP} fa uso degli \emph{Status Attributes} per la valutazione di una richiesta in input.
Questi attributi sono modificabili tramite alcune azioni dette \emph{PepAction}, come l'operazione di somma \emph{add(Attribute,int)}
oppure l'assegnamento di una data con \emph{setDate(Attribute,date)}.
Sono definiti nel \ac{PAS} come descritto nell'esempio precedente con questa struttura:
\lstinputlisting[firstline = 8, lastline = 8,language = FACPL]{code/cap4/charliePAS}\label{lst:esempio_sintassiPAS2}
\par Il \ac{PEP} oltre a verificare che le PepAction siano eseguite, con l'aggiunta delle \emph{Obbligation Check}, ha il
controllo nella gestione delle richieste continuative. Se nella policy è presente una obbligazione che fa parte del nuovo tipo,
il \ac{PEP} ha il compito di controllare che le \emph{Expression} contenute nell'\emph{Obligation Check} siano vere.\par
Nell'esempio precedente l'obbligazione è definita con:
\lstinputlisting[firstline = 8, lastline = 8,language = FACPL]{code/cap4/charlieRead}\label{lst:esempio_sintassiObl}
il \ac{PEP} deve verificare quindi che l'azione richiesta sia una "read" e che la risorsa sia "contabilita.xlsx".
Se il controllo dà esito positivo la richiesta è subito accettata, se la richiesta non passa una delle verifiche si continuerà
il processo di valutazione ripartendo dal \ac{PDP} e svolgendo tutti i passi.\par
Le \emph{Obligation Check} possono essere permanenti, limitate temporalmente oppure limitate sul numero di richieste
accettate.\par
\begin{description}[labelindent=5pt,style=multiline,leftmargin=3.5cm]
  \item[Permanenti]Se sono del primo tipo e la verifica sulle \emph{Expression} è sempre vera, il \ac{PEP} continuerà a gestire le
  richieste senza il controllo del \ac{PDP}. La gestione cambia solo quando una richiesta non è conforme a quelle accettate.\par
  \item[Scadenza sul tempo]Se sono del secondo tipo, anche se la verifica sulle \emph{Expression} è vera, quando il tempo limite è stato superato,
  la gestione delle richieste cambia e si riprende il processo valutativo completo.
  \item[Scadenza sul numero di richieste]Il terzo tipo di \emph{Obligation Check} è simile al secondo, con l'unica differenza che il limite non è temporale,
  ma sul numero di richieste elaborate.
\end{description}

\section{Esempi di Usage Control}
\label{sec:Esempi}
In questa sezione si mostrano gli esempio presentati in Sezione~\ref{subs:es_UC} utilizzando le nuove funzionalità.
\subsection{Accessi in lettura e scrittura di file}
\label{sub:RW_Code_Sec}
Nel Codice \ref{lst:esempio_RWP} si mostra la policy relativa all'esempio \ref{subs:Accessi in lettura e scrittura di file}.
Nel sistema ci sono due utenti: Alice e Bob. Alice fa parte del gruppo degli Administrator,
mentre Bob appartine ad un gruppo di default non specificato. Solo gli amministratori possono scrivere,
però tutti possono leggere file appartenenti ad un insieme precedentemente specificato come si mostra in linea 28.

Se l'Administrator Alice fa una richiesta di scrittura, ha l'obbligo di cambiare il booleano \emph{isWriting} in \texttt{true} (linea 14).
Se finisce di scrivere invece deve fare una richiesta di \emph{stopWrite} e assegnare \texttt{false} all'attributo (linea 25).
Ovviamente se non sta scrivendo non può richiedere la fine dell'azione quindi \emph{stopWrite} è accettabile solo quando
isWriting è \texttt{true} (linea 21).
\lstinputlisting[language = FACPL,stepnumber=2,lastline=26, caption = {Policy per esempio lettura e scrittura }\label{lst:esempio_RWP}]{code/cap4/checkReadWritePolicy}
\vspace{1em}
Tutti gli utenti possono leggere i file solo se nessuno sta scrivendo su uno di questi (linea 32) e
se i file appartengono ad un determinato insieme (linea 28). In \emph{Read\_Policy} si specifica che i
file debbano essere "thesis.tex" oppure "facpl.pdf", inoltre viene utilizzato il nuovo tipo di obbligation alla linea 35.
Il \ac{PAS} è differente dalla versione usata in Sezione~\ref{sec:Esempio_FACPL} solo nella definizione dello status
in cui si inizializza isWriting a \texttt{false} (linea 45).
\clearpage
\lstinputlisting[language = FACPL,stepnumber=2,firstline=27,firstnumber=27]{code/cap4/checkReadWritePolicy}\label{lst:esempio_RWP2}
Di seguito, nel Codice \ref{lst:esempio_RWR} e \ref{lst:esempio_RWR2}, si mostrano otto richieste per descrivere tutti i possibili scenari
insieme ai risultati di valutazione. \par
Le prime tre \emph{request} richiedono una lettura sia da parte di Bob che di Alice. Queste tre richieste sono simili tra loro,
differiscono solo per i file richiesti e tutte saranno accettatte perché rispettano Read\_Policy. La \emph{Request1} anche
se essenzialmente uguale alle altre due passa per un processo di valutazione diverso. Infatti solo la prima richiesta
sarà valutata dal \ac{PDP} e dal \ac{PEP} mentre le altre solo da quest'ultimo.
Nella prima si controlla che l'azione sia "read" (linea 31 Codice \ref{lst:esempio_RWP}),
il file sia nell'insieme \{"thesis.tex","facpl.pdf"\} (linea 28) e che il Boolean isWriting sia falso (linea 32).
Nella seconda e nella terza invece il \ac{PEP} verifica solo che l'azione sia una lettura e che il file richiesto sia valido.
I controlli del \ac{PEP} sono espressi nella linea 35 della policy. Si fa notare la differenza dei tempi di valutazione
tra la \emph{Request1} e le altre due.
\clearpage
\lstinputlisting[firstline = 1, lastline = 15,language = FACPL,stepnumber=2, caption = {Richieste per esempio lettura e scrittura }\label{lst:esempio_RWR}]{code/cap4/checkReadWriteRequests}
\begin{verbatim}
  ---------------------------------------------------
  REQUEST N: 1
  REQUEST: read_request Bob
   Decision=
   PERMIT
  time 53ms
  ---------------------------------------------------
  REQUEST N: 2
  REQUEST: read_request Bob
   Decision=
   PERMIT
  time 6ms
  ---------------------------------------------------
  REQUEST N: 3
  REQUEST: read_request Alice
   Decision=
   PERMIT
  time 9ms
\end{verbatim}
La \emph{Request4} interrompe il processo valutativo ridotto in quanto l'azione è una "write" e si ritorna alla valutazione
completa includendo i controlli del \ac{PDP}. La \emph{Request5} e la \emph{Request6} non vengono accettate perché l'amministratore
sta scrivendo sul file e solo successivamente alla richiesta di "stopWrite" di Alice, Bob può leggere il documento.
La \emph{Request8} è una richiesta di lettura, ed essendo successiva ad un'altra read (\emph{Request7}), si adotterà
nuovamente il processo di valutazione ridotto.\par
\lstinputlisting[firstline = 16, language = FACPL, caption = {Richieste per esempio lettura e scrittura }\label{lst:esempio_RWR2}]{code/cap4/checkReadWriteRequests}
\begin{verbatim}
  ---------------------------------------------------
  REQUEST N: 4
  REQUEST: write_request Alice
   Decision=
   PERMIT
  time 47ms
  ---------------------------------------------------
  REQUEST N: 5
  REQUEST: read_request Bob
   Decision=
   DENY
  time 25ms
  ---------------------------------------------------
  REQUEST N: 6
  REQUEST: stop_write_request Alice
   Decision=
   PERMIT
  time 44ms
  ---------------------------------------------------
  REQUEST N: 7
  REQUEST: read_request Bob
   Decision=
   PERMIT
  time 32ms
  ---------------------------------------------------
  REQUEST N: 8
  REQUEST: read_request Alice
   Decision=
   PERMIT
  time 3ms
\end{verbatim}
In Tabella \ref{tab:risultati_1} si riassumono i risultati della valutazione, il tipo di processo valutativo
(\ac{PDP}+\ac{PEP} per quello completo e \ac{PEP} per quello ridotto) e i tempi di valutazione.
\begin{table}[h]
\centering
\footnotesize
\caption{Risultati della valutazione}
\label{tab:risultati_1}
\begin{tabular}{cccc}
\multicolumn{1}{l}{} & \textbf{Risultato} & \textbf{PV} & \textbf{Tempo} \\ \hline
\textbf{Richiesta 1} & \textit{PERMIT} & \textit{PDP+PEP} & \textit{53ms} \\ \hline
\textbf{Richiesta 2} & \textit{PERMIT} & \textit{PEP} & \textit{6ms} \\ \hline
\textbf{Richiesta 3} & \textit{PERMIT} & \textit{PEP} & \textit{9ms} \\ \hline
\textbf{Richiesta 4} & \textit{PERMIT} & \textit{PDP+PEP} & \textit{47ms} \\ \hline
\textbf{Richiesta 5} & \textit{DENY}   & \textit{PDP+PEP} & \textit{25ms} \\ \hline
\textbf{Richiesta 6} & \textit{PERMIT} & \textit{PDP+PEP} & \textit{44ms} \\ \hline
\textbf{Richiesta 7} & \textit{PERMIT} & \textit{PDP+PEP} & \textit{32ms} \\ \hline
\textbf{Richiesta 8} & \textit{PERMIT} & \textit{PEP} & \textit{3ms} \\ \hline
\end{tabular}
\end{table}
\subsection{Servizio di streaming}
\label{sub:Stream_Code_Sec}
L'esempio seguente è simile al primo, ma in questo caso si fa uso anche delle \emph{Obligation Check} limitate dal tempo.
Ci sono due utenti. Alice è un utente \emph{Premium} e può ascoltare le canzoni senza limiti una volta fatto il login. Bob
è un utente standard, ha lo stesso vincolo di login di Alice, ma dopo un certo lasso di tempo non può più fare richieste
di ascolto e dovrà sentire la pubblicità prima di poter ascoltare di nuovo un brano.
Per implementare la "listen" senza limiti di Alice e quella limitata dal tempo di Bob si usano Obligation Check differenti.

Nel Codice \ref{lst:esempio_SP1} si mostrano le policy per il "login" e per le "listen" degli utenti Premium.
Nella policy di login si controlla che la password della richiesta sia quella salvata nel sistema (linea 8) e se è giusta
si assegna all'attributo \emph{loginAlice} "PREMIUM" e a \emph{streamingAlice} \texttt{true} (linee 11-12).
Gli ascolti sono associati ad una obbligazione persistente (linea 22).
Gli utenti possono quindi richiedere i brani senza essere limitati per tutta la durata del login.
La prima richiesta di "listen" di Alice è valutata dal \ac{PDP} e dal \ac{PEP} eseguendo i controlli riportati
nelle linee 16-19. Per le richieste successive, valutate solo dal \ac{PEP}, si controlla invece l'obbligazione in linea 22.
\lstinputlisting[language = FACPL,lastline=23,stepnumber=2, caption = {Policy Login e Listen Premium}\label{lst:esempio_SP1}]{code/cap4/checkStreamingPolicy}
Il login per gli utenti standard è simile a quello dei premium e cambia soltanto nell'assegnamento delle variabili (linee 30-31).
Gli ascolti invece sono associati ad una obbligazione limitata dal tempo. Gli utenti standard possono ascoltare
le canzoni per quindici minuti di fila, come specificato nell'Obligation in linea 43, ma una volta esaurito il tempo
devono ascoltare la pubblicità. Alla prima richiesta di "listen" si assegna al booleano associato al vincolo sulla
pubblicità \emph{commercialsBob} \texttt{true} affinché, una volta esaurito il tempo, l'utente Bob debba fare una richiesta
di "listenCommercials" per richiedere nuovamente un'ascolto.
Anche in questo caso le richieste d'ascolto sono gestite interamente dal \ac{PEP} a partire dalla
seconda richiesta consecutiva.
\lstinputlisting[language = FACPL, firstline=25,stepnumber=2,firstnumber=23, caption = {Policy Login, Listen e Commercials Standard}\label{lst:esempio_SP2}]{code/cap4/checkStreamingPolicy}
Le prime due \emph{request} non sono accettate perché non è possibile richiedere gli ascolti prima di fare un login.
\lstinputlisting[language = FACPL,lastline=8, caption = {Richieste di ascolto prima del login }\label{lst:esempio_SR1}]{code/cap4/checkStreamingRequests}
\begin{verbatim}
  ---------------------------------------------------
  REQUEST N: 1
  REQUEST: listen alice
   Decision=
   DENY
  time 55ms
  ---------------------------------------------------
  REQUEST N: 2
  REQUEST: listen bob
   Decision=
   DENY
  time 39ms
\end{verbatim}
Dopo \emph{Request3} e \emph{Request4}, dove si usa un attributo password per validare l'accesso, i due utenti possono ascoltare i brani.
\lstinputlisting[language = FACPL,firstline=9,lastline=18,firstnumber=9, caption = {Richieste di login }\label{lst:esempio_SR2}]{code/cap4/checkStreamingRequests}
\begin{verbatim}
  ---------------------------------------------------
  REQUEST N: 3
  REQUEST: login alice
   Decision=
   PERMIT
  time 24ms
  ---------------------------------------------------
  REQUEST N: 4
  REQUEST: login bob
   Decision=
   PERMIT
  time 38ms
\end{verbatim}
Dalla \emph{Request5} alla \emph{Request8} gli utenti richiedono degli ascolti e sono tutti autorizzati. Si fa notare anche
in questo esempio la differenza tra i tempi di valutazione delle varie richieste. La \emph{Request6} e la \emph{Request8}
impiegano meno tempo perché viene adottato il processo valutativo ridotto.
\lstinputlisting[language = FACPL,firstline=19,lastline=34,firstnumber=19, caption = {Richieste di listen}\label{lst:esempio_SR2}]{code/cap4/checkStreamingRequests}
\begin{verbatim}
  ---------------------------------------------------
  REQUEST N: 5
  REQUEST: listen alice
   Decision=
   PERMIT
  time 34ms
  ---------------------------------------------------
  REQUEST N: 6
  REQUEST: listen alice
   Decision=
   PERMIT
  time 8ms
  ---------------------------------------------------
  REQUEST N: 7
  REQUEST: listen bob
   Decision=
   PERMIT
  time 60ms
  ---------------------------------------------------
  REQUEST N: 8
  REQUEST: listen bob
   Decision=
   PERMIT
  time 7ms
\end{verbatim}
Si suppone che dopo tre richieste di ascolto, Bob alla \emph{Request9} abbia superato il limite di tempo di quindici minuti.
Tutte le sue possibili richieste di ascolto sono rifiutate finché non richiede di sentire la pubblicità.
\lstinputlisting[language = FACPL,firstline=35,lastline=42,firstnumber=35, caption = {Richieste di listen Bob}\label{lst:esempio_SR2}]{code/cap4/checkStreamingRequests}
\begin{verbatim}
  ---------------------------------------------------
  REQUEST N: 9
  REQUEST: listen bob
   Decision=
   DENY
  time 29ms
  ---------------------------------------------------
  REQUEST N: 10
  REQUEST: listen bob
   Decision=
   DENY
  time 27ms
\end{verbatim}
Dopo una richiesta di "listenCommercials" Bob ha di nuovo altri quindici minuti di ascolto possibili.
\lstinputlisting[language = FACPL,firstline=43,firstnumber=43, caption = {Richiesta di ascolto della pubblicità di Bob}\label{lst:esempio_SR2}]{code/cap4/checkStreamingRequests}
\begin{verbatim}
  REQUEST N: 11
  REQUEST: pubblicità bob
   Decision=
   PERMIT
  time 33ms
  ---------------------------------------------------
  REQUEST N: 12
  REQUEST: listen bob
   Decision=
   PERMIT
  time 36ms
\end{verbatim}
Anche in questo esempio si propone una tabella in cui si riassumono i risultati della valutazione.
\begin{table}[h]
\centering
\footnotesize
\caption{Risultati della valutazione}
\label{tab:risultati_2}
\begin{tabular}{cccc}
\multicolumn{1}{l}{}  & \textbf{Risultato}  & \textbf{PV}       & \textbf{Tempo}  \\ \hline
\textbf{Richiesta 1}  & \textit{DENY}       & \textit{PDP+PEP}  & \textit{55ms}   \\ \hline
\textbf{Richiesta 2}  & \textit{DENY}       & \textit{PDP+PEP}  & \textit{39ms}   \\ \hline
\textbf{Richiesta 3}  & \textit{PERMIT}     & \textit{PDP+PEP}  & \textit{24ms}   \\ \hline
\textbf{Richiesta 4}  & \textit{PERMIT}     & \textit{PDP+PEP}  & \textit{38ms}   \\ \hline
\textbf{Richiesta 5}  & \textit{PERMIT}     & \textit{PDP+PEP}  & \textit{34ms}   \\ \hline
\textbf{Richiesta 6}  & \textit{PERMIT}     & \textit{PEP}      & \textit{8ms}    \\ \hline
\textbf{Richiesta 7}  & \textit{PERMIT}     & \textit{PDP+PEP}  & \textit{60ms}   \\ \hline
\textbf{Richiesta 8}  & \textit{PERMIT}     & \textit{PEP}      & \textit{7ms}    \\ \hline
\textbf{Richiesta 9}  & \textit{DENY}       & \textit{PDP+PEP}  & \textit{29ms}   \\ \hline
\textbf{Richiesta 10} & \textit{DENY}       & \textit{PDP+PEP}  & \textit{27ms}   \\ \hline
\textbf{Richiesta 11} & \textit{PERMIT}     & \textit{PDP+PEP}  & \textit{33ms}   \\ \hline
\textbf{Richiesta 12} & \textit{PERMIT}     & \textit{PDP+PEP}  & \textit{36ms}   \\ \hline
\end{tabular}
\end{table}

In questi due esempi si è mostrato come sia flessibile il processo di valutazione. Il sistema
riesce a gestire a runtime il controllo degli accessi e adatta il processo di verifica basandosi sul
tipo di richieste. Inoltre sono evidenti i miglioramenti a livello prestazionale. Richieste ripetute
sono gestite in modo più efficiente e se valutate solo dal \ac{PEP} la risposta impiega meno della metà del tempo
per essere restituita.
