\myChapter{Introduzione}
Dalla loro nascita i sistemi informatici hanno avuto il ruolo di gestore di dati.
Il tipo di queste informazioni ha reso necessario l'utilizzo di un sistema che le proteggesse, infatti i
dati più sensibili, se diffusi senza una valida autorizzazione, possono arrecare ad esempio danni economici
ad una società o nuocere anche gli utenti nel privato.
Per far fronte al problema sono stati sviluppati dei modelli per il controllo degli accessi. Tuttavia la quantità di dati
e la complessità dei sistemi moderni ha mostrato i limiti delle tecnologie concepite e questo fatto ha portato inevitabilmente
allo sviluppo di un nuovo modello: lo \ac{UCON}

In questa tesi si descrive un'estensione basata su UCON per \ac{FACPL}.
Il problema principale da risolvere è la staticità del processo di valutazione del linguaggio.
Tutte le richieste di accesso ricevono una risposta attraverso un sistema di verifica sempre uguale e i controlli
risultano ridondati e in alcuni casi inutili.
Nei capitoli successivi si mostrano i passi che hanno portato allo sviluppo di una nuova gestione delle richieste
in grado di rendere dinamico il processo valutativo.
\vspace{5mm}
\newline
Il resto del documento è così strutturato:
\begin{itemize}
  \renewcommand\labelitemi{--}
  \item Nel capitolo~\ref{chap:Access Control e Usage Control}
  si introducono vari modelli di Access Control, si descrive lo \ac{UCON} e si mostrano due casi di studio basati sul nuovo modello.
  \item Nel capitolo~\ref{chap:Formal Access Control Policy Language}
  si descrive la sintassi e la semantica di \ac{FACPL}, il processo di valutazione e un esempio utilizzando il linguaggio.
  \item Nel capitolo~\ref{chap:Usage Control in FACPL}
  si mostra l'estensione di \ac{FACPL} per migliorare la gestione delle richieste.
  \item Nel capitolo~\ref{chap:Estensione della libreria FACPL}
  si riportano le modifiche alla libreria Java su cui è basato il linguaggio per adattarlo al nuovo processo valutativo.
  \item Nel capitolo~\ref{chap:Conclusioni}
  si riassume il lavoro svolto e si presentato dei possibili sviluppi futuri.
\end{itemize}
