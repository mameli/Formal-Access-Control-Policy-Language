\myChapter{Introduzione}
I sistemi informatici, fin dalla loro nascita, sono stati utilizzati per la gestione di dati.
Il tipo di informazioni gestite ha reso necessario la creazione di un sistema che le proteggesse, infatti i
dati più sensibili, se diffusi senza una valida autorizzazione, possono arrecare ad esempio danni economici
ad una società o nuocere anche gli utenti nel privato.
Per far fronte al problema sono stati sviluppati dei modelli per il controllo degli accessi. Tuttavia la quantità di dati
e la complessità dei sistemi moderni ha mostrato i limiti delle tecnologie concepite e questo fatto ha portato inevitabilmente
allo sviluppo di un nuovo approccio: lo \acl{UCON}.

I ricercatori dell'università di San Antonio, Jaehong Park e Ravi Sandhu, hanno descritto il modello
\ac{UCON}~\cite{ucon} che consente di ottenere controlli continuativi di accesso e di basare le valutazioni di una richiesta
sul comportamento passato.

In questa tesi si descrive un'estensione basata su UCON per \ac{FACPL}.
\ac{FACPL} è un linguaggio basato su \ac{XACML} ed è supportato da una libreria Java. Questo viene utilizzato per lo
sviluppo di sistemi basati sul modello per controllo degli accessi \acl{PBAC}.\par
Il problema principale da risolvere è la staticità del processo di valutazione del linguaggio.
Tutte le richieste di accesso ricevono una risposta attraverso un sistema di verifica che esegue sempre
tutti i controlli. Non modificando mai i passi della valutazione, vari controlli per le richieste potrebbero
risultare ridondati e in alcuni casi inutili. Le richieste che non modificano i dati
o lo stato del sistema, come ad esempio le letture su file, passano per lo stesso processo di valutazione
di tutte le altre richieste e il sistema dei controlli risulta eccessivamente complesso per l'autorizzazione di azioni semplici.
Per questo motivo è stato ideato un monitor in grado di scegliere a runtime il processo di verifica delle richieste.\par
Nella prima parte dello sviluppo è stata modificata la sintassi del linguaggio per adeguarla ad una nuova
gestione delle richieste, in seguito
è stata sviluppata un'estensione della libreria Java che supporta \ac{FACPL} per modificare
il processo valutativo così da garantire controlli di accesso continuativi.
Nei capitoli successivi si mostrano i passi che hanno portato allo sviluppo della nuova gestione delle richieste
in grado di rendere dinamico il processo di valutazione.
\vspace{5mm}
\newline
Il resto del documento è così strutturato:
\begin{itemize}
  \renewcommand\labelitemi{--}
  \item Nel Capitolo~\ref{chap:Access Control e Usage Control}
  si introducono vari modelli di Access Control, si descrive \ac{UCON} e si mostrano due casi di studio basati sul
  controllo continuativo degli accessi.
  \item Nel Capitolo~\ref{chap:Formal Access Control Policy Language}
  si descrive la sintassi e la semantica di \ac{FACPL}, il processo di valutazione e un esempio di utilizzo del linguaggio.
  \item Nel Capitolo~\ref{chap:Usage Control in FACPL}
  si mostra l'estensione di \ac{FACPL} al fine di supportare controlli continativi di accesso.
  \item Nel Capitolo~\ref{chap:Estensione della libreria FACPL}
  si riportano le modifiche alla libreria Java che supporta \ac{FACPL} per adattarlo al nuovo processo valutativo.
  \item Nel Capitolo~\ref{chap:Conclusioni}
  si riassume il lavoro svolto e si presentano alcuni possibili sviluppi futuri.
\end{itemize}
