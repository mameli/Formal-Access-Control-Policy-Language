\myChapter{Conclusioni}
\label{chap:Conclusioni}
In questa tesi è stata sviluppata un'estensione del linguaggio \ac{FACPL} per migliorare la gestione delle richieste di accesso.
Nella prima parte sono state descritte le strutture su cui si è basato lo sviluppo. Sono stati introdotti i primi modelli
dell'Access Control partendo dal più semplice \ac{ACL} arrivando fino a \ac{PBAC}. In seguito è stato mostrato lo Usage Control
descrivendo il modello UCON\textsubscript{ABC} ideato da Jaehong Park e Ravi Sandhu.

Successivamente è stato esaminato il linguaggio \ac{FACPL} su cui si è basata l'implementazione del monitor a runtime.
\MakeUppercase{è} stata mostrata la sintassi, la semantica e il processo di valutazione del linguaggio.
Utilizzando un esempio di politica, sono stati evidenziati i limiti e le strutture mancanti per
il supporto al controllo continuativo degli accessi.

Per eliminare i controlli superflui delle verifiche di autorizzazione, è stata modificata la sintassi di \ac{FACPL}
e il processo valutativo.
Alle modifiche apportate sono seguite le implementazioni di nuove strutture nella libreria Java
che supporta \ac{FACPL}.
Infine, sulla base degli esempi di Usage Control proposti, sono stati mostrati i miglioramenti nell'uso delle
risorse computative.

Il lavoro di estensione della libreria è stato svolto insieme al mio collega Federico Schipani. Nella tesi del
mio collega si parla di come è stato possibile esprimere in \ac{FACPL} politiche di controllo degli accessi basate
sul comportamento passato. Le modifiche riguardano principalmente il \ac{PDP} (descritto nel Capitolo~\ref{chap:Formal Access Control Policy Language})
e consistono nella creazione di uno stato del sistema. In questa tesi invece si modifica il \ac{PEP}
(descritto ugualmente nel Capitolo~\ref{chap:Formal Access Control Policy Language}
e nel Capitolo~\ref{chap:Usage Control in FACPL} e \ref{chap:Estensione della libreria FACPL} nella versione modificata),
si implementa un nuovo tipo di obbligazione e si mostrano i miglioramenti dei tempi di valutazione delle richieste.
\section{Sviluppi Futuri}
\label{sec:Sviluppi Futuri}
In questo documento sono stati esposti due esempi di Usage Control, uno si basa sul numero di richieste, l'altro sul tempo.
La libreria di \ac{FACPL} è semplice da estendere ed è facile pensare a nuovi possibili controlli su cui
le autorizzazioni possono basarsi. Si potrebbe associare le verifiche della richiesta alla posizione
geografica dell'utente, oppure alla congestione della rete aggiungendo semplicemente la gestione di un nuovo parametro nella
creazione delle obbligazioni e le rispettive funzioni che ne controllano il valore.
