\chapter{Access Control e Usage Control}
\label{chap:Access Control e Usage Control}
\section{Controllo degli accessi}
\label{sec:Controllo degli accessi}
La protezione dei dati ha determinato la necessità di creare strumenti per il controllo degli accessi che potevano eliminare
,o almeno limitare, i rischi derivati dalla perdita delle informazioni.\par
Nel corso del tempo si sono sviluppati alcuni modelli per i sistemi del controllo degli accessi. A seconda delle esigenze
sono stati adottati numerosi tipi di tecnologie\cite{NISTACM}. Nelle sezioni successive se ne presentano alcune.
\subsection{Access Control List}
\label{sub:ACL}
Access Control List(ACL) è stato creato agli inizi degli anni settanta per la necessità di un controllo degli
accessi sui sistemi multiutente.
Utilizza una lista di utenti con annesse le possibili azioni autorizzate. Il modello è molto semplice,
ma ha molte limitazioni. Quando nel sistema ci sono numerosi utenti o risorse, la quantità di dati da verificare diventa
difficile da gestire. Questo può portare a errori di assegnazione di autorizzazioni e ad un eccesivo numero di controlli
necessari per un singolo accesso.
\subsection{Role Based Access Control}
\label{sub:RBAC}
Role Based Access Control (RBAC) è l' evoluzione di ACL. In questo modello vengono introdotti i \emph{ruoli}. Più utenti
possono avere lo stesso ruolo e quindi avere a disposizione le tutte risorse connesse a questo. Il modello
diventa scalabile e più facile da gestire, inoltre si possono anche creare delle gerarchie per facilitare l' assegnamento di
risorse in base alla classificazione dell' utente.
\MyFigure{img/RBAC}{RBAC}{0.7}
\subsection{Attribute Based Access Control}
\label{sub:ABAC}
Attribute Based Access Control (ABAC) si basa sull'utilizzo di attributi associati all'utente, all'azione o al contesto
della richiesta. La valutazione di una autorizzazione diventa più specifica e le regole sono più precise per ogni risorsa.
Questo tipo di modello non è utilizzato nei sistemi operativi,dove ACL e RBAC sono i modelli più diffusi, ma è sviluppatto
spesso a livello applicativo. Il problema fondamentale di questo paradigma è che le regole non sono uniformi e se il numero
di risorse è consistente, la gestione di queste diventa complicata. Il modello Policy Based Access Control cerca di
risolvere i difetti di ABAC.
\MyFigure{img/abac.png}{ABAC}{0.7}
\subsection{Policy Based Access Control}
\label{sub:PBAC}
Policy Based Access Control riorganizza il modello ABAC per semplificare la gestione delle regole.
Il sistema si basa su \emph{politiche} che non sono altro che insiemi di \emph{regole}. A ogni regola è associato un attributo
che l'utente deve avere, e ogni politica valuta tutte le regole nel suo insieme per creare la risposta sull'autorizzazione. Anche
le politiche possono essere messe insieme per creare gruppi di politiche, in questo modo il sistema diventa scalabile e di
più facile utilizzo.\par
Per costruire un sistema di controllo degli accessi basato sul modello PBAC è necessario l'utilizzo di
un linguaggio adatto allo scopo. L'organizzazione OASIS (Organization for the Advancement of
 Structured Information Standards) ha creato il linguaggio eXtensible Access
Control Markup Language (XACML) che è diventato lo standard per lo sviluppo di un sistema costruito sul modello PBAC.
\section{Usage Control}
\label{sec:Usage Control}
Dopo quaranta anni di studi sul controllo degli accessi i modelli sviluppati si sono consolidati e sono largamente utilizzati
su sistemi operativi o applicazioni. Tuttavia la complessità e la varietà degli ambienti informatici moderni va oltre i limiti
dei modelli creati.\par
Il termine Usage Control (UCON) è stato ripreso da Jaehong Park e Ravi Sandhu per creare il modello UCON\textsubscript{ABC}\cite{ucon},
questo è una generalizzazione dell'Access Control che include obbligazioni, condizioni sull'utilizzo,
controlli continuativi e mutabilità. Comprende e migliora i modelli di controllo di accesso tradizionali, quali Trust
Management (TM) e Digital Rights Management (DRM) aggiungendo la gestione di attributi variabili e la continuità nella valutazione
delle decisioni per l'accesso. Il modello UCON\textsubscript{ABC} estende i controlli sull'accesso tradizionali ed è composto da otto componenti fondamentali.
Queste sono subjects,subject attributes, objects,objects attributes, rights, authorizations, obligations e conditions.\par
\MyFigure{img/ucon.png}{UCON}{1.1}
I \emph{Subjects} sono entità a cui si associano degli attributi e hanno o esercitano \emph{Rights} sugli \emph{Objects}. Possiamo per semplicità
associare i Subjects ad un singolo individuo umano.\par
Gli \emph{Objects} sono insiemi di entità su cui i \emph{Subjects} possono avere dei \emph{Rights}, questi possono essere usati o vi si può
fare accesso. Possono essere associati ad esempio a un libro, o a una qualsiasi risorsa.\par
I \emph{Rights} sono i privilegi che i \emph{Subjects} hanno o esercitano sugli \emph{Objects}.\par
I tre fattori Authorizations, oBligations e Conditions(da cui prende anche il
nome il modello UCON\textsubscript{ABC}) sono predicati funzionali che devono essere valutati per le decisioni sull'uso.
I tradizionali Access Controls utilizzano solo le Authorizations per il processo di decisione, Obligations e Conditions
sono i nuovi componenti che entrano a far parte della valutazione.\par
Le \emph{Authorizations} devono valutare la decisione sull'uso. Queste danno un responso positivo
o negativo a seconda che la domanda di un Subject sia accettata o meno.\par
Le \emph{Obligation} verificano i requisiti obbligatori che un Subject deve eseguire prima o durante l'utilizzo di una risorsa.\par
Infine le \emph{Condition} restituiscono true o false in base alle variabili dell'ambiente o allo stato del sistema.\par
Il processo di decisione è diviso in tre fasi\cite{SurveyUsageControl}:Before usage(pre), Ongoing usage(on) e After usage.
\MyFigure{img/uconFase.png}{Fasi del processo di decisione}{1}
La valutazione della prima parte inizia da una richiesta e non ha differenze con il processo valutativo dell'Access Control.
Nella seconda invece si utilizzano i nuovi predicati introdotti ed è in questa parte che si affermano i controlli continuativi,
le obbligazioni e le condizioni sull'utilizzo.\par
L'ultima parte varia in base agli eventi delle fasi precedenti. Ad esempio se il Subject che ha richiesto un accesso ha violato
una policy oltre al non aver ricevuto l'autorizzazione potrebbe anche essere ammonito e il sistema potrebbe non accettare più
nessuna sua richiesta.

\subsubsection{Esempi di Usage Control}
\label{subs:Esempi}
