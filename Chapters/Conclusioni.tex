\myChapter{Conclusioni}
\label{chap:Conclusioni}
In questa tesi è stata sviluppata un'estensione del linguaggio \ac{FACPL} per migliorare la gestione delle richieste di accesso.
Nella prima parte sono state descritte le strutture su cui si è basato lo sviluppo. Sono stati introdotti i primi modelli
dell'Access Control partendo dal più semplice \ac{ACL} arrivando fino a \ac{PBAC}. In seguito è stato mostrato lo Usage Control
descrivendo il nuovo modello UCON\textsubscript{ABC} creato da Jaehong Park e Ravi Sandhu.

Successivamente è stato esaminato il linguaggio \ac{FACPL} su cui si è basato implementazione del monitor a runtime.
\MakeUppercase{è} stata mostrata la sintassi, la semantica e il processo di valutazione per presentare i lati positivi
e soprattutto i limiti del linguaggio su cui si è lavorato per migliorare il controllo continuativo degli accessi.

Sono state aggiunte nuove produzioni alla sintassi ed è stata modificata la risposta dell'autorizzazione per poter
essere utilizzata nel nuovo processo valutativo, in cui si elimina i controlli superflui per un determinato tipo di richieste.
Alle modifiche apportate sulla sintassi sono seguite le implementazioni di nuove strutture nella libreria Java su cui si
basa \ac{FACPL}. Infine sono stati proposti esempi per presentare le operazioni possibili e i miglioramenti nell'uso delle
risorse computative.

Il lavoro di estensione della libreria è stato svolto insieme al mio collega Federico Schipani. Nella tesi del
mio collega si parla di come è stato possibile esprimere in FACPL politiche di controllo degli accessi basate
sul comportamento passato. Le modifiche riguardano principalmente il \ac{PDP} (descritto nel Capitolo~\ref{chap:Formal Access Control Policy Language})
e consistono nella creazione di uno stato del sistema. In questa tesi invece si modifica il \ac{PEP}
(descritto ugualmente nel Capitolo~\ref{chap:Formal Access Control Policy Language}
e nel Capitolo~\ref{chap:Usage Control in FACPL} e \ref{chap:Estensione della libreria FACPL} nella versione modificata)
e si implementa un nuovo tipo di obbligazione.
\section{Sviluppi Futuri}
\label{sec:Sviluppi Futuri}
In questo documento sono stati esposti due esempi, in uno si usa il numero di richieste, nell'altro il tempo.
La libreria di \ac{FACPL} è semplice da estendere ed è facile pensare a nuovi possibili controlli su cui
le autorizzazioni possono basarsi. Si potrebbe associare le verifiche della richiesta alla posizione
geografica dell'utente, oppure alla congestione della rete aggiungendo semplicemente la gestione di un nuovo parametro nella
creazione delle obbligazioni e le rispettive funzioni che ne controllano il valore.
