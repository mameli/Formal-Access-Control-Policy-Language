%--------------------------------------------------------------
% thesis.tex
%--------------------------------------------------------------
% Corso di Laurea in Informatica
% http://if.dsi.unifi.it/
% @Facolt\`a di Scienze Matematiche, Fisiche e Naturali
% @Universit\`a degli Studi di Firenze
%--------------------------------------------------------------
% - template for the main file of Informatica@Unifi Thesis
% - based on Classic Thesis Style Copyright (C) 2008
%   Andr\'e Miede http://www.miede.de
%--------------------------------------------------------------
\documentclass[twoside,openright,titlepage,fleqn,
	headinclude,12pt,a4paper,BCOR5mm,footinclude]{scrbook}
%--------------------------------------------------------------
\newcommand{\myItalianTitle}{Estensione del linguaggio FACPL per esprimere politiche di gestione dell'utilizzo continuativo delle risorse di un sistema di calcolo\xspace}
\newcommand{\myEnglishTitle}{Extension of language FACPL To use access control policies based on continuative use of resouces\xspace}
% use the right myDegree option
\newcommand{\myDegree}{Corso di Laurea in Informatica\xspace}
%\newcommand{\myDegree}{
	%Corso di Laurea Specialistica in Scienze e Tecnologie
	%dell'Informazione\xspace}
\newcommand{\myName}{Filippo Mameli\xspace}
\newcommand{\myProf}{Rosario Pugliese\xspace}
\newcommand{\myOtherProf}{Andrea Margheri\xspace}
\newcommand{\mySupervisor}{\xspace}
\newcommand{\myFaculty}{Scuola di Scienze Matematiche, Fisiche e Naturali\xspace}
\newcommand{\myUni}{\protect{Universit\`a degli Studi di Firenze}\xspace}
\newcommand{\myLocation}{Firenze\xspace}
\newcommand{\myTime}{Anno Accademico 2015-2016\xspace}
\newcommand{\myVersion}{Version 0.1\xspace}
%--------------------------------------------------------------
\usepackage[italian]{babel}
\usepackage[utf8x]{inputenc}
\usepackage[T1]{fontenc}
\usepackage[square,numbers]{natbib}
\usepackage[fleqn]{amsmath}
\usepackage{ellipsis}
\usepackage{listings}
\usepackage{subfig}
\usepackage{caption}
\usepackage{appendix}
\usepackage{siunitx}
\usepackage{float}

%--------------------------------------------------------------
\usepackage{dia-classicthesis-ldpkg}
%--------------------------------------------------------------
% Options for classicthesis.sty:
% tocaligned eulerchapternumbers drafting linedheaders
% listsseparated subfig nochapters beramono eulermath parts
% minionpro pdfspacing
\usepackage[eulerchapternumbers,linedheaders,subfig,beramono,eulermath,
parts]{classicthesis}
%--------------------------------------------------------------
\newlength{\abcd} % for ab..z string length calculation
% how all the floats will be aligned
\newcommand{\myfloatalign}{\centering}
\setlength{\extrarowheight}{3pt} % increase table row height
\captionsetup{format=hang,font=small}
%--------------------------------------------------------------
% Comandi personali
%--------------------------------------------------------------
\usepackage{verbatimbox}
\newcommand{\MyFigure}[3]{
	\begin{figure}[h]

		\centering
		\includegraphics[width=#3\linewidth]{#1}
		\caption{ #2 }\label{fig:#1}

	\end{figure}
}

\newcommand{\MyFig}[4]{
	\begin{figure}[#4]

		\centering
		\includegraphics[width=#3\linewidth]{#1}
		\caption{ #2 }\label{#1}

	\end{figure}
}

\newcommand{\statusattribute}{\textit{Status Attribute}}
\newcommand{\status}{\textit{Status}}


\newcommand{\sr}[1]{#1}

\newcommand{\define}{\triangleq}



\newcommand{\modif}[1]{{\color{red}#1}}

%---------------------------------------
%ACRONYM
%---------------------------------------
\newcommand{\xacml}{\ac{XACML}}
\newcommand{\facpl}{\ac{FACPL}}
\newcommand{\pbac}{\ac{PBAC}}
\newcommand{\rbac}{\ac{RBAC}}

\newcommand{\pdp}{\ac{PDP}}
\newcommand{\pep}{\ac{PEP}}
\newcommand{\pap}{\ac{PAP}}
\newcommand{\pip}{\ac{PIP}}
\newcommand{\pr}{\ac{PR}}

\newcommand{\dsl}{\ac{DSL}}
\newcommand{\ide}{\ac{IDE}}

%%%%%%%%%%%%%%%%%%%%%%%%%%%%%
%SYNTAX
%%%%%%%%%%%%%%%%%%%%%%%%%%%%%
%-----------------------------------
%Algorithm
%-----------------------------------

\newcommand{\denyOver}{\x{d}\textrm{-}\x{over}_{\delta}}
\newcommand{\permitOver}{\x{p}\textrm{-}\x{over}_{\delta}}
\newcommand{\permitUnless}{\x{p}\textrm{-}\x{unless}\textrm{-}\x{d}_{\delta}}
\newcommand{\denyUnless}{\x{d}\textrm{-}\x{unless}\textrm{-}\x{p}_{\delta}}
\newcommand{\onlyOneApp}{\x{one}\textrm{-}\x{app}_{\delta}}
\newcommand{\firstApp}{\x{first}\textrm{-}\x{app}_{\delta}}
\newcommand{\weakCon}{\x{weak}\textrm{-}\x{con}_{\delta}}
\newcommand{\strongCon}{\x{strong}\textrm{-}\x{con}_{\delta}}


\newcommand{\denyOverO}[1]{\x{d}\textrm{-}\x{over}_{\x{#1}}}
\newcommand{\permitOverO}[1]{\x{p}\textrm{-}\x{over}_{\x{#1}}}
\newcommand{\permitUnlessO}[1]{\x{p}\textrm{-}\x{unless}\textrm{-}\x{d}_{\x{#1}}}
\newcommand{\denyUnlessO}[1]{\x{d}\textrm{-}\x{unless}\textrm{-}\x{p}_{\x{#1}}}
\newcommand{\onlyOneAppO}[1]{\x{one}\textrm{-}\x{app}_{\x{#1}}}
\newcommand{\firstAppO}[1]{\x{first}\textrm{-}\x{app}_{\x{#1}}}
\newcommand{\weakConO}[1]{\x{weak}\textrm{-}\x{con}_{\x{#1}}}
\newcommand{\strongConO}[1]{\x{strong}\textrm{-}\x{con}_{\x{#1}}}

%Obligation Types
\newcommand{\obM}{\modif{\x{M}}}
\newcommand{\obO}{\modif{\x{O}}}


%PEP
\newcommand{\based}{\x{base}}
\newcommand{\debugAlg}{\x{debug}}
\newcommand{\denyBiased}{\x{deny}\textrm{-}\x{biased}}
\newcommand{\permitBiased}{\x{permit}\textrm{-}\x{biased}}

%-------------------------------------
%Syntax formatting, comparison functions, examples
%-------------------------------------

\newcommand{\alice}{\x{Alice}}
\newcommand{\bob}{\x{Bob}}

\newcommand{\polSet}[4]{{\bf{\{}}#1{\,}\x{target:}\,#2 {?}\, \x{policies:}#3^{+} {\,}\x{obl:}#4^{*}\, {\bf{\}}}}
\newcommand{\polSetE}[4]{{\bf{\{}}#1 \, \x{target:}\,#2 \, \x{policies:}#3 \, \x{obl:}#4 \, {\bf{\}}}}
\newcommand{\pol}[4]{{\bf{\langle}}#1{\,}\x{target:}\,#2 {?}{\,}\, \x{rules:}#3^{+} {\,}\x{obl:}#4^{*}{\,}{\bf{\rangle}}}
\newcommand{\polE}[4]{{\bf{\langle}}#1{\,}\x{target:}\,#2 {\,}\, \x{rules:}#3 {\,}\x{obl:}#4 {\,}{\bf{\rangle}}}

\newcommand{\pdpPol}[2]{\{ #1 \,  \, #2 \}}
\newcommand{\obl}[1]{\, PepAction( #1^* ) }

\newcommand{\streq}{\x{equal}}
\newcommand{\exprOp}{\x{eop}}
\newcommand{\exprOperator}{\x{op}}

\newcommand{\attribute}[2]{( #1 , #2 )}

%\newcommand{\mEl}{m}
%\newcommand{\mEval}[2]{#2 \models #1}

\newcommand{\x}[1]{{\sf #1}}


%---------------------------------------
% Separator
%---------------------------------------

\newcommand{\set}[1]{\{#1\}} % brackets for sets
\newcommand{\Sep}{\ \mid\ }

\newcommand{\policySetBegin}{{\bf{\{}}}
\newcommand{\policyBegin}{{\bf{\{}}}
\newcommand{\policySep}{\,\bf{\ }\,}
\newcommand{\policiesBegin}{\x{policies:}\,}
\newcommand{\targetBegin}{\x{target:}\,}
\newcommand{\targetEnd}{\,{\bf{\ }}}
\newcommand{\policiesEnd}{\,{\bf{\ }}}
%\newcommand{\rulesBegin}{\x{rules:}\,}
%\newcommand{\rulesEnd}{\,{\bf{\ }}}
\newcommand{\policyEnd}{{\bf{\}}}}
\newcommand{\policySetEnd}{{\bf{\}}}}
%\newcommand{\conditionBegin}{\x{condition:}\,}
%\newcommand{\conditionEnd}{\,{\bf{\ }}}
\newcommand{\oblsBegin}{\x{obl:}\,}
\newcommand{\oblsEnd}{\bf{\ }\,}
\newcommand{\oblBegin}{{\bf{[}}}
\newcommand{\oblEnd}{{\bf{]}}}

\newcommand{\match}[3]{#1{\bf(}#2{\bf,}#3{\bf)}}
\newcommand{\ruleOpt}[1]{{\bf{(}}#1{\bf{)}}}
\newcommand{\oblOpt}[1]{{\bf{(}}#1{\bf{)}}}

\newcommand*\lfrac[2]{{}_{#1}\!\backslash\!^{#2}} % table combining algorithm

%------------------------------
%Decisions & values
%------------------------------
\newcommand{\permit}{\x{permit}}
\newcommand{\deny}{\x{deny}}
\newcommand{\notApp}{\x{not}\textrm{-}\x{app}}
\newcommand{\indet}{\x{indet}}
\newcommand{\excpt}{\perp}
\newcommand{\err}{\x{error}}

\newcommand{\indetD}{\x{indetD}}
\newcommand{\indetP}{\x{indetP}}
\newcommand{\indetDP}{\x{indetDP}}

\newcommand{\true}{\x{true}}
\newcommand{\false}{\x{false}}

%%%%%%%%%%%%%%%%%%%%%%%
%SEMANTICS
%%%%%%%%%%%%%%%%%%%%%%%

%Auxiliary Semantic Functions and notations
\newcommand{\pepSemR}[1]{(\!( #1 )\!)}
\newcommand{\concat}{{\bullet}}
\newcommand{\subobeff}[2]{#1\!\!\mid_{#2}} %sub sequence of obligations

%Name of Semantic Function
\newcommand{\denSemF}[1]{\mathcal{ #1 }}

%Semantic Brakets
%\newcommand{\denSem}[2]{[\![ #1 ]\!]_{#2}}
\newcommand{\denSem}[2]{[\![ #1 ]\!] #2}

%Named Semantics Functions
\newcommand{\policySem}[2]{\denSemF{P}\denSem{#1}{#2}}
\newcommand{\algSem}[2]{\denSemF{A}\denSem{#1}{#2}}
\newcommand{\exprSem}[2]{\denSemF{E}\denSem{#1}{#2}}
\newcommand{\oblSem}[2]{\denSemF{O}\denSem{#1}{#2}}
\newcommand{\oblSemS}[2]{\denSemF{OS}\denSem{#1}{#2}}
\newcommand{\pdpSem}[2]{\denSemF{P}dp\denSem{#1}{#2}}
\newcommand{\pepSem}[2]{\denSemF{E}A\denSem{#1}{#2}}
\newcommand{\pasSem}[1]{\denSemF{P}as\denSem{#1}{}}
\newcommand{\reqSemS}[2]{\denSemF{R}\denSem{#1}{#2}}  %two arguments, \name as second argument
\newcommand{\reqSem}[1]{\denSemF{R}\denSem{#1}{}} %used for PAS semantics

%Generic Element of each Syntactic Category
\newcommand{\expr}{\mathit{expr}}
\newcommand{\effect}{\mathit{e}}
\newcommand{\ob}{\mathit{o}}
\newcommand{\obType}{\mathit{t}}
\newcommand{\fo}{\mathit{f\!o}}
\newcommand{\foS}{\mathit{f\!o}^*}
\newcommand{\rSyntax}{\mathit{req}} %element of syntactic category Request
\newcommand{\req}{\mathit{r}} % element of functional representation of Request
\newcommand{\policy}{\mathit{p}}
\newcommand{\algSyntax}{\mathit{a}}
\newcommand{\pdpRes}{\mathit{res}}
\newcommand{\dec}{\mathit{dec}} %decision
\newcommand{\double}{\mathit{d}} %double
\newcommand{\extVal}{\mathit{w}} % Obligations and table of expression semantics
\newcommand{\enfAlg}{\mathit{ea}}
\newcommand{\pdpSyntax}{\mathit{pdp}}
\newcommand{\name}{\mathit{n}}
\newcommand{\val}{\mathit{v}}
\newcommand{\pepAction}{\mathit{pepAct}}

\newcommand{\algNT}{\mathit{Alg}} %used in syntax section
\newcommand{\algName}{\mathsf{alg}} % used in combining algorithms semantic section

%Semantics of Combiing Algorithms
\newcommand{\alg}[1]{\mathsf{alg}_{#1}}
\newcommand{\algD}{\alg{\delta}}
\newcommand{\algOp}{\otimes \mathsf{alg}}
\newcommand{\algOpAlg}[1]{\otimes #1}

\newcommand{\all}{\x{all}}
\newcommand{\greedy}{\x{greedy}}
\newcommand{\isFinal}[2]{\mathit{isFinal}_{#1}({#2})}
\newcommand{\isFinalPred}[1]{\mathit{isFinal}_{#1}}

%--------------------------------------------------------------
% Layout setting
%--------------------------------------------------------------
\usepackage{geometry}
\geometry{
	a4paper,
	ignoremp,
	bindingoffset = 1cm,
	textwidth     = 13.5cm,
	textheight    = 21.5cm,
	lmargin       = 3.5cm, % left margin
	tmargin       = 4cm    % top margin
}

\lstset{
  	frame=tb,
	language=Matlab,
  	aboveskip=3mm,
  	belowskip=3mm,
  	showstringspaces=false,
  	columns=flexible,
  	basicstyle={\small\ttfamily},
  	numbers=none,
  	breaklines=true,
  	breakatwhitespace=true,
  	tabsize=3
}

\definecolor{eclipsePurple}{RGB}{127,0,85}
\definecolor{dkgreen}{rgb}{0,.6,0}
\definecolor{brightpink}{rgb}{1.0, 0.0, 0.5}
\definecolor{burgundy}{rgb}{0.5, 0.0, 0.13}

\lstdefinelanguage{FACPL} {
  % list of keywords
	keywords = [1] {
		deny-biased, permit-biased
	},
  	morekeywords = [2] {
		target, policies, obl
	},
	keywords = [3] {
		addStatus, subStatus, equal, flagstatus,
		less-than, greater-than
	},
	keywords = [4] {
		Policy, PolicySet, Rule
	},
	keywords = [5] {
		permit, deny
	},
  %style
	keywordstyle = [1]{
		[3]\color{dkgreen}
	},
	keywordstyle = [2]{
		\color{burgundy}
	},
	keywordstyle = [3]{
		\color{dkgreen}
	},
	keywordstyle = [4]{
		\color{orange}
	},
	keywordstyle = [5]{
		\color{brightpink}
	},
	sensitive=false,
	morecomment=[l]{//},
	morecomment=[s]{/*}{*/},
	morestring=[b]",
	morestring=[b]',
	stringstyle=\color{blue}\ttfamily
}

%--------------------------------------------------------------
\begin{document}
\frenchspacing
\raggedbottom
\pagenumbering{roman}
\pagestyle{plain}
%--------------------------------------------------------------
% Frontmatter
%--------------------------------------------------------------
%--------------------------------------------------------------
% titlepage.tex (use thesis.tex as main file)
%--------------------------------------------------------------
\begin{titlepage}
	\begin{center}
   	\large
      \hfill
      \vfill
      \begingroup
         \includegraphics[scale=0.15]{logo/LOGO}\\
%			\spacedallcaps{\myUni} \\
			\myFaculty \\
			\myDegree \\
			\vspace{0.5cm}
         \vspace{0.5cm}
         Tesi di Laurea
      \endgroup
      \vfill
      \begingroup
      	\color{Maroon}\spacedallcaps{\myItalianTitle} \\ $\ $\\
      	\spacedallcaps{\myEnglishTitle} \\
	\bigskip
      \endgroup
      \spacedlowsmallcaps{\myName}
      \vfill
      \vfill
      Relatore: \emph{Rosario Pugliese}\\
      Correlatore: \emph{Andrea Margheri}\\
      \vfill
      \vfill
      \myTime
      \vfill
	\end{center}
\end{titlepage}
%--------------------------------------------------------------
% back titlepage
%--------------------------------------------------------------
   \newpage
	\thispagestyle{empty}
	\hfill
	\vfill
	\noindent\myName:
	\textit{\myItalianTitle,}
	\myDegree, \textcopyright\ \myTime
%--------------------------------------------------------------
% back titlepage end
%--------------------------------------------------------------

\pagestyle{scrheadings}
%--------------------------------------------------------------
% Mainmatter
%--------------------------------------------------------------
\pagenumbering{arabic}
% use \cleardoublepage here to avoid problems with pdfbookmark
%\include{intro} % use \myChapter command instead of \chapter
\tableofcontents
%\listoffigures
\cleardoublepage
%\thispagestyle{empty}
%\begin{flushright}
%\null\vspace{\stretch {1}}
%\emph{"Inserire citazione" \break --- Inserire autore citazione} \vspace{\stretch{2}}\null
%\end{flushright}
%\cleardoublepage
\myChapter{Introduzione}
Dalla loro nascita i sistemi informatici hanno avuto il ruolo di gestore di dati.
Il tipo di queste informazioni ha reso necessario l'utilizzo di un sistema che le proteggesse, infatti i
dati più sensibili, se diffusi senza una valida autorizzazione, possono arrecare ad esempio danni economici
ad una società o nuocere anche gli utenti nel privato.
Per far fronte al problema sono stati sviluppati dei modelli per il controllo degli accessi. Tuttavia la quantità di dati
e la complessità dei sistemi moderni ha mostrato i limiti delle tecnologie concepite e questo fatto ha portato inevitabilmente
allo sviluppo di un nuovo modello: lo \ac{UCON}

In questa tesi si descrive un'estensione basata su UCON per \ac{FACPL}.
Il problema principale da risolvere è la staticità del processo di valutazione del linguaggio.
Tutte le richieste di accesso ricevono una risposta attraverso un sistema di verifica sempre uguale e i controlli
risultano ridondati e in alcuni casi inutili.
Nei capitoli successivi si mostrano i passi che hanno portato allo sviluppo di una nuova gestione delle richieste
in grado di rendere dinamico il processo valutativo.
\vspace{5mm}
\newline
Il resto del documento è così strutturato:
\begin{itemize}
  \renewcommand\labelitemi{--}
  \item Nel capitolo~\ref{chap:Access Control e Usage Control}
  si introducono vari modelli di Access Control, si descrive lo \ac{UCON} e si mostrano due casi di studio basati sul nuovo modello.
  \item Nel capitolo~\ref{chap:Formal Access Control Policy Language}
  si descrive la sintassi e la semantica di \ac{FACPL}, il processo di valutazione e un esempio utilizzando il linguaggio.
  \item Nel capitolo~\ref{chap:Usage Control in FACPL}
  si mostra l'estensione di \ac{FACPL} per migliorare la gestione delle richieste.
  \item Nel capitolo~\ref{chap:Estensione della libreria FACPL}
  si riportano le modifiche alla libreria Java su cui è basato il linguaggio per adattarlo al nuovo processo valutativo.
  \item Nel capitolo~\ref{chap:Conclusioni}
  si riassume il lavoro svolto e si presentato dei possibili sviluppi futuri.
\end{itemize}

\myChapter{Access Control e Usage Control}
\label{chap:Access Control e Usage Control}
\section{Controllo degli accessi}
\label{sec:Controllo degli accessi}
La protezione dei dati ha determinato la necessità di creare strumenti per il controllo degli accessi che potevano eliminare
,o almeno limitare, i rischi derivati dalla perdita delle informazioni.\par
Nel corso del tempo si sono sviluppati alcuni modelli per i sistemi del controllo degli accessi. A seconda delle esigenze
sono stati adottati numerosi tipi di tecnologie\cite{NISTACM}. Nelle sezioni successive se ne presentano alcune.
\subsection{Access Control List}
\label{sub:ACL}
Access Control List(ACL) è stato creato agli inizi degli anni settanta per la necessità di un controllo degli
accessi sui sistemi multiutente.
Utilizza una lista di utenti con annesse le possibili azioni autorizzate. Il modello è molto semplice,
ma ha molte limitazioni. Quando nel sistema ci sono numerosi utenti o risorse, la quantità di dati da verificare diventa
difficile da gestire. Questo può portare a errori di assegnazione di autorizzazioni e ad un eccesivo numero di controlli
necessari per un singolo accesso.
\subsection{Role Based Access Control}
\label{sub:RBAC}
Role Based Access Control (RBAC) è l' evoluzione di ACL. In questo modello vengono introdotti i \emph{ruoli}. Più utenti
possono avere lo stesso ruolo e quindi avere a disposizione le tutte risorse connesse a questo. Il modello
diventa scalabile e più facile da gestire, inoltre si possono anche creare delle gerarchie per facilitare l' assegnamento di
risorse in base alla classificazione dell' utente.
\MyFigure{img/RBAC}{RBAC}{0.7}
\subsection{Attribute Based Access Control}
\label{sub:ABAC}
Attribute Based Access Control (ABAC) si basa sull'utilizzo di attributi associati all'utente, all'azione o al contesto
della richiesta. La valutazione di una autorizzazione diventa più specifica e le regole sono più precise per ogni risorsa.
Questo tipo di modello non è utilizzato nei sistemi operativi,dove ACL e RBAC sono i modelli più diffusi, ma è sviluppatto
spesso a livello applicativo. Il problema fondamentale di questo paradigma è che le regole non sono uniformi e se il numero
di risorse è consistente, la gestione di queste diventa complicata. Il modello Policy Based Access Control cerca di
risolvere i difetti di ABAC.
\MyFigure{img/abac.png}{ABAC}{0.7}
\subsection{Policy Based Access Control}
\label{sub:PBAC}
Policy Based Access Control riorganizza il modello ABAC per semplificare la gestione delle regole.
Il sistema si basa su \emph{politiche} che non sono altro che insiemi di \emph{regole}. A ogni regola è associato un attributo
che l'utente deve avere, e ogni politica valuta tutte le regole nel suo insieme per creare la risposta sull'autorizzazione. Anche
le politiche possono essere messe insieme per creare gruppi di politiche, in questo modo il sistema diventa scalabile e di
più facile utilizzo.\par
Per costruire un sistema di controllo degli accessi basato sul modello PBAC è necessario l'utilizzo di
un linguaggio adatto allo scopo. L'organizzazione OASIS (Organization for the Advancement of
 Structured Information Standards) ha creato il linguaggio eXtensible Access
Control Markup Language (XACML) che è diventato lo standard per lo sviluppo di un sistema costruito sul modello PBAC.
\section{Usage Control}
\label{sec:Usage_Control}
Dopo quaranta anni di studi sul controllo degli accessi i modelli sviluppati si sono consolidati e sono largamente utilizzati
su sistemi operativi o applicazioni. Tuttavia la complessità e la varietà degli ambienti informatici moderni va oltre i limiti
dei modelli creati.\par
Il termine Usage Control (UCON) è stato ripreso da Jaehong Park e Ravi Sandhu per creare il modello UCON\textsubscript{ABC}\cite{ucon},
questo è una generalizzazione dell'Access Control che include obbligazioni, condizioni sull'utilizzo,
controlli continuativi e mutabilità. Comprende e migliora i modelli di controllo di accesso tradizionali, quali Trust
Management (TM) e Digital Rights Management (DRM) aggiungendo la gestione di attributi variabili e la continuità nella valutazione
delle decisioni per l'accesso. Il modello UCON\textsubscript{ABC} estende i controlli sull'accesso tradizionali ed è composto da otto componenti fondamentali.
Queste sono subjects,subject attributes, objects,objects attributes, rights, authorizations, obligations e conditions.\par
\MyFigure{img/ucon.png}{UCON}{1.1}
I \emph{Subjects} sono entità a cui si associano degli attributi e hanno o esercitano \emph{Rights} sugli \emph{Objects}. Possiamo per semplicità
associare i Subjects ad un singolo individuo umano.\par
Gli \emph{Objects} sono insiemi di entità su cui i \emph{Subjects} possono avere dei \emph{Rights}, questi possono essere usati o vi si può
fare accesso. Possono essere associati ad esempio a un libro, o a una qualsiasi risorsa.\par
I \emph{Rights} sono i privilegi che i \emph{Subjects} hanno o esercitano sugli \emph{Objects}.\par
I tre fattori Authorizations, oBligations e Conditions(da cui prende anche il
nome il modello UCON\textsubscript{ABC}) sono predicati funzionali che devono essere valutati per le decisioni sull'uso.
I tradizionali Access Controls utilizzano solo le Authorizations per il processo di decisione, Obligations e Conditions
sono i nuovi componenti che entrano a far parte della valutazione.\par
Le \emph{Authorizations} devono valutare la decisione sull'uso. Queste danno un responso positivo
o negativo a seconda che la domanda di un Subject sia accettata o meno.\par
Le \emph{Obligation} verificano i requisiti obbligatori che un Subject deve eseguire prima o durante l'utilizzo di una risorsa.\par
Infine le \emph{Condition} restituiscono true o false in base alle variabili dell'ambiente o allo stato del sistema.\par
Il processo di decisione è diviso in tre fasi\cite{SurveyUsageControl}:Before usage(pre), Ongoing usage(on) e After usage.
\MyFigure{img/uconFase.png}{Fasi del processo di decisione}{1}
La valutazione della prima parte inizia da una richiesta e non ha differenze con il processo valutativo dell'Access Control.
Nella seconda invece si utilizzano i nuovi predicati introdotti ed è in questa parte che si affermano i controlli continuativi,
le obbligazioni e le condizioni sull'utilizzo.\par
L'ultima parte varia in base agli eventi delle fasi precedenti. Ad esempio se il Subject che ha richiesto un accesso ha violato
una policy oltre al non aver ricevuto l'autorizzazione potrebbe anche essere ammonito e il sistema potrebbe non accettare più
nessuna sua richiesta.

\subsection{Esempi di Usage Control}
\label{subs:es_UC}
Si propongono adesso due esempio in cui si utilizza lo Usage Control
\subsubsection*{Accessi in lettura e scrittura di file}
\label{subs:Accessi in lettura e scrittura di file}
In un sistema ci sono alcuni file e gli utenti sono divisi in gruppi, si suppone inoltre che le richieste di lettura
siano più comuni rispetto a quelle di scrittura.\par Tutti possono leggere i file, ma solo
gli utenti nel gruppo degli amministratori possono modificare un elemento. Se uno degli amministratori sta scrivendo
su un file, nessuno può leggerlo o apportare modifiche nello stesso momento. Una volta finita la scrittua si potrà sbloccare
il file e renderlo di nuovo disponibile a tutti.\par
Letture ripetute, su uno stesso insieme di file, vengono gestite in modo efficiente.
\subsubsection{Servizio di streaming}
\label{subs:Servizio di streaming}
In un sistema gli utenti, dopo aver fatto il login, possono fare una richiesta di ascolto di un brano. I clienti
sono divisi in due tipologie. Chi effettua il login come utente premium può fare richieste di ascolto senza nessuna restrizione.
I clienti stardard hanno invece un tempo limite e una volta esaurito devono ascoltare la pubblità prima di poter fare
un'altra richiestad di ascolto\par
In questo caso si assumono le richieste di ascolto più numerose rispetto ai login o ai logout.

\myChapter{Formal Access Control Policy Language }
\label{chap:Formal Access Control Policy Language}
Il linguaggio \ac{FACPL} (fakpol) è stato creato come alternativa a \ac{XACML}. Come accennato nel in \ref{sec:PBAC},
l'organizzazione \ac{OASIS} ha ideato il linguaggio XACML per sviluppare sistemi basati sul modello \ac{PBAC}.\par
\ac{XML} è utilizzato da \ac{XACML} per definire le sue politiche. Questo linguaggio di markup è molto usato per lo scambio
di dati tra sistemi, ma rende le politiche difficili da comprendere. Infatti anche le regole più semplici sono prolisse
e questo rende la lettura problematica per un utente.\par
\ac{FACPL} nasce dalle idee di \ac{XACML} e ha l'obiettivo di semplificare la scrittura di politiche e di definire un framework
costruito sopra basi formali solide, in modo da permettere agli sviluppatori di specificare e verificare automaticamente
delle proprietà.

In questo capitolo si delinea prima di tutto il processo di valutazione in \ref{sec:Il processo di valutazione Facpl}
poi mostrano le componenti della sistema in \ref{sub:Componenti del sistema} e il loro significato in \ref{sec:sem_fpl}. Infine si
mostrerà in \ref{sec:Esempio_FACPL} con un esempio l'utilizzo di \ac{FACPL} .
\section{Il processo di valutazione}
\label{sec:Il processo di valutazione Facpl}
In questa sezione si espone la sequenza di azioni che il sistema compie affinchè una richiesta in input sia valutata.
\MyFigure{img/evaluationProcess.png}{Processo di valutazione}{1}
In figura \ref{fig:img/evaluationProcess.png} si mostrano i passi che vengono eseguiti.
Le componenti chiave del processo sono tre:
\begin{itemize}
  \renewcommand\labelitemi{--}
  \item \ac{PR}
  \item \ac{PDP}
  \item \ac{PEP}
\end{itemize}
Si assume che un insieme di risorse sia in coppia con le Policy, e che queste definiscano le credenziali
necessarie per ottenere l'accesso. Il \ac{PR} contiene le Policy e le rende disponibili al \ac{PDP} (Step 1), il
quale decide se l'accesso viene garantito o meno. \par
Quando la richiesta è ricevuta dal \ac{PEP} (Step 2), le sue credenziali vengono codificate in una sequenza
di attributi(una coppia nome-valore) per poi utilizzare quest'ultima per creare la richiesta \ac{FACPL}(Step 3).\par
Il Context Handler aggiunge attributi dell'ambiente (come la temperatura esterna) alla richiesta e
poi invierà questa al \ac{PDP} (Step 4).\par
Il processo di autorizzazione del \ac{PDP} restituisce una \emph{risposta} verificando che gli attributi, che possono far
parte della richiesta o del contesto, siano conformi ai controlli delle policy (Step 5-8). La \emph{risposta} del \ac{PDP}
contiene una \emph{decisione} e una possibile \emph{obbligazione}(Step 9-10).\par
La \emph{decisione} può essere di quattro tipi:
\begin{itemize}
  \renewcommand\labelitemi{--}
  \item Permit
  \item Deny
  \item Not-applicable
  \item Indeterminate
\end{itemize}
I primi due tipi indicano rispettivamente richiesta accettata e richiesta non accettata. Se viene restituito Not-applicable
non ci sono policy su cui si poteva valutare la decisione, se ci sono altri errori la risposta è Indeterminate.
Le \ePolicy gestiscono automaticamente il risultato Indeterminate combinandolo con le altre decisioni
a seconda della strategia del \emph{Combining Algorithm} associato alla Policy stessa.\par
Le \eObligations sono azioni addizionali che devono essere eseguite dopo la restituzione di una decisione.
Solitamente corrispondono all'aggiornamento di un file, l'invio di un messaggio o l'esecuzione di un comando.
Il \ac{PEP} ha compito di controllare lo svolgimento delle \eObligations tramite l'\emph{obligation service} (Step 11-12).
Il processo di \emph{enforcement} eseguito dal \ac{PEP} determina l'\emph{enforced decision} in base al risultato delle obbligazioni.
Questa decisione può differire da quella del \ac{PDP} e corrisponde alla valutazione finale di tutto il processo.


\section{Componenti del sistema}
\label{sub:Componenti del sistema}
Nella tabella \ref{tab:facpl_syntax} si mostra la sintassi di \ac{FACPL}. Questa è data da una grammatica di tipo \ac{EBNF} dove il simbolo ? indica
elementi opzionali, * sequenze (anche vuote) e + sequenze non vuote.\par
Al livello più alto troviamo il termine  \ac{PAS} che comprende le specifiche del \ac{PEP} e del \ac{PDP}.\par
Il \ac{PEP} è definito con un \emph{enforcing algorithm} che sarà applicato per stabilire quali sono le decisioni che devono
passare al processo di \emph{enforcement}.\par
Il \ac{PDP} è definito come una sequenza di \emph{Policy}\textsuperscript{+} e da un algoritmo \emph{Alg} per combinare i
risultati delle valutazione delle policy.\par
\begin{table}[h]
\centering
\scriptsize
\caption{Sintassi di FACPL}
\hrule
$
\begin{array}{@{\,}r@{\ \ }r@{\ }r@{\ \ }l@{\ }}
{\textbf{Policy Authorisation Systems}} &
\mathit{PAS} & ::= & ( \,  \x{pep:} \, \mathit{EnfAlg}\ \ \x{pdp:}\, \mathit{PDP} \, )
\\[.2cm]
{\textbf{Enforcement algorithms}} &
\mathit{EnfAlg}
& ::= & \based \Sep \denyBiased \Sep \permitBiased
\\[.2cm]
{\textbf{Policy Decision Points}} &
\mathit{PDP} & ::= & \pdpPol{\algNT\ }{\x{policies:} \, \mathit{Policy}^{+}}
\\[.2cm]
{\textbf{Combining algorithms}} &
\algNT & ::= & \permitOver \Sep \denyOver \Sep \denyUnless \Sep \permitUnless \\
&& \mid &
\firstApp \Sep \onlyOneApp \Sep \weakCon \Sep \strongCon
\\[.2cm]
{\textbf{Policies}} &
\mathit{Policy} & ::= &
\ruleOpt{\mathit{Effect}\ \ \x{target:} \, Expr\ \ \x{obl:} \, \mathit{Obligation}^{*} \, } \\
&& \mid &
\{ \algNT\ \ \x{target:} \, Expr\ \
\x{policies:} \, \mathit{Policy}^{+} \ \ \x{obl:} \, \mathit{Obligation}^{*} \, \}
\\[.2cm]
{\textbf{Effects}} &
\mathit{Effect} & ::= & \permit \Sep \deny
\\[.2cm]
{\textbf{Obligations}} &
\mathit{Obligation} & ::= & [ \, \mathit{Effect} \ \ \mathit{ObType} \ \ \obl{Expr} \, ]
\\[.2cm]
{\textbf{Obligation Types}} &
\mathit{ObType} & ::= & M \Sep O
\\[.4cm]
\textbf{Expressions}&
\mathit{Expr} & ::= &
\mathit{Name} \Sep \mathit{Value}  \\
& & \mid &\x{and(\mathit{Expr}, \mathit{Expr})} \Sep \x{or(\mathit{Expr}, \mathit{Expr})} \Sep \x{not(\mathit{Expr})} \\
& & \mid &
 \x{equal(\mathit{Expr},\mathit{Expr})}  \Sep \x{in}(\mathit{Expr}, \mathit{Expr}) \\
& & \mid & \x{greater}\textrm{-}\x{than(\mathit{Expr},\mathit{Expr})} \Sep \x{add(\mathit{Expr} ,\mathit{Expr} )}\\
& & \mid & \x{subtract(\mathit{Expr} ,\mathit{Expr} )} \Sep \x{divide(\mathit{Expr} ,\mathit{Expr} )}\\
& & \mid & \x{multiply(\mathit{Expr} ,\mathit{Expr} )} \\[.2cm]
%
\textbf{Attribute Names} &
\mathit{Name} & ::= & \mathit{Identifier}/\mathit{Identifier} \\[.2cm]
%
\textbf{Literal Values} &
\mathit{Value} & ::= & \x{true} \mid \x{false} \mid \mathit{Double} \mid \mathit{String} \mid \mathit{Date}
\\[.4cm]
{\textbf{Requests}} &
\mathit{Request} & ::= & {\attribute{\mathit{Name}}{\mathit{Value}}}^{+}
\\[.1cm]
\end{array}
$
\hrule
\label{tab:facpl_syntax}
\end{table}
{\label{Sintassi di FACPL}}
Una \ePolicy può essere una \eRule semplice oppure un \ePolicySet cioè un insieme di rule o altri
policy set. In questo modo si possono creare regole singole, ma anche gerarchie di regole.\par
Un \ePolicySet è definito da un \etarget che indica l'insieme di richieste di accesso al quale la policy
viene applicata, da una lista di \eObligations cioè le azioni opzionali o obbligatoriche da eseguire, da una
sequenza di \ePolicy e da un algoritmo per la combinazione.\par
Una \eRule  è specificata da un \emph{effect} , che può essere permit o deny , da un \etarget
e da una lista di \eObligations(che può essere anche vuota). \par
Le \emph{Expressions} sono costituite da \emph{attribute names} e da valori letterali(per esempio booleani, stringhe, date).\par
Un \emph{attribute name} indica il valore di un attributo. Questo può essere contenuto in una richiesta o nel contesto. La
struttura di un attribute name è della forma \emph{Identifier}/\emph{Identifier}. Dove il primo elemento indica la categoria
e il secondo il nome dell'attributo. Per esempio Name / ID rappresenta il valore di un attributo ID di categoria Name.\par
Un \emph{combining algorithm} ha lo scopo di risolvere conflitti delle decisioni date dalle valutazioni delle policy.
Ad esempio permit-overrides assegna la precedenza alle decisioni con effetto permit rispetto a quelle di tipo deny.\par
Una \eObligation è definita da un effetto, da un tipo (M per obbligatorio e O per opzionale) e da un'azione e le relative
espressioni come argomento.\par

Una richiesta consiste in una sequenza di \emph{attribute} organizzati in categorie.\par
La risposta ad una richiesta \ac{FACPL} è scritta utilizzando la sintassi ausiliaria riportata in tabella \ref{tab:facpl_context_syntax}.
La valutazione in due passi descritta in \ref{sec:Il processo di valutazione Facpl} produce due tipi differenti di risposte:
\begin{itemize}
  \renewcommand\labelitemi{--}
  \item \emph{\ac{PDP} Response}
  \item \emph{Decisions}
\end{itemize}
La prima, nel caso in cui la decisione sia permit o deny, si associa a una sequenza (anche vuota) di fulfilled obligations.\par
Una \emph{Fulfilled Obligation} è una coppia formata da un tipo e da un'azione con i rispettivi argomenti risultato della
valutazione del \ac{PDP}.
\input{extraTex/Facpl_Syntax2}

\section{Semantica}
\label{sec:sem_fpl}
Si presenta adesso informalmente il processo di autorizzazione del \ac{PDP} e poi quello di enforcement del \ac{PEP}.\par
Quando il \ac{PDP} riceve una richiesta di accesso, valuta la richiesta in base alle \epolicy disponibili poi determina
il risultato unendo le decisioni restituite dalle policy con l'utilizzo dei \emph{combining algorithms}.\par
La valutazione di una policy inizia dalla verifica dell'applicabilità della richiesta, che è compiuta valutando
l'espressione definita dal \etarget. Ci sono due casi:
\begin{description}
  \item[\normalfont\bfseries{\MakeUppercase{L}a verifica dà un risultato positivo.}] \hfill \\
        Nel caso in cui ci siano \emph{rules}, l'effetto della regola viene restituito.
        Nel caso di \ePolicySet, il risultato è ottenuto dalla valutazione delle \epolicy contenute e dalla combinazione di queste,
        tramite l'algoritmo specificato dal \ac{PDP}. In entrambi i casi in seguito si procederà al \emph{fulfilment} delle
        \eobligation da parte del \ac{PEP}.
  \item[\normalfont\bfseries{\MakeUppercase{L}a verifica dà un risultato negativo.}] \hfill \\
        Nel caso in cui la valutazione sia determinata \efalse, viene restituito
        \texttt{not-app}. Nel caso di \texttt{ERROR} o di un valore non booleano, si restituisce \texttt{indet}.
\end{description}
Valutare le espressioni corrisponde ad applicare gli operatori, a determinare le occorrenze degli \emph{attribute names} e
a ricavarne il valore associato.\par
Se questo non è possibile, ad esempio l'attibuto è mancante e non può essere recuperato dal \emph{context handler}, si
restituisce il valore speciale \texttt{BOTTOM}. Questo valore è usato per implementare strategie diverse per gestire
la mancanza di un attributo. \texttt{BOTTOM} è trattato da \ac{FACPL} in modo simile a \texttt{false}, si gestisce
così gli attributi mancanti senza generare errori.\par
Gli operatori tengono conto degli argomenti e dei valori speciali come \texttt{BOTTOM} e \texttt{ERROR}.
Se gli argomenti sono \texttt{true} oppure \texttt{false} gli operatori sono applicati in modo regolare.
Se c'è un argomento \texttt{BOTTOM} e non ci sono \texttt{ERROR} si restituisce \texttt{BOTTOM}, \texttt{ERROR} altrimenti.
Gli operatori \eand e \eor trattano diversamente i valori speciali. Specificatamente, \eand restituisce \texttt{true} se entrambi
gli operandi sono \etrue, \texttt{false} se almeno uno è \efalse, \texttt{BOTTOM} se almeno uno è \texttt{BOTTOM} e nessun altro è
\texttt{false} o \texttt{ERROR}, \texttt{ERROR} negli altri casi. L'operatore \eor è il duale di \eand e ha priorità minore.
L'operatore unario \emph{not} cambia i valori di \texttt{true} e \texttt{false}, ma non quelli di \texttt{BOTTOM} e \texttt{ERROR}.\par
La valutazione di una \epolicy termina con il \emph{fulfilment} di tutte le \emph{Obligations}. Questo consiste nel valutare
tutti gli argomenti dell'azione corrispondente all'\emph{obligation}. All'occorrenza di un errore, la decisione della \epolicy
sarà modificata in \texttt{indet}.
Negli altri casi la decisione non cambierà e sarà quella del \ac{PDP} prima del \emph{fulfillment}.\par
Per valutare gli insiemi di \epolicy si devono applicare dei \emph{combining algorithm} specifici. Data una sequenza di
policy in input gli algoritmi stabiliscono una sequenza di valutazioni per le \epolicy date. Si propone in seguito l'algoritmo
\emph{permit-overrides} come esempio.
\begin{description}[labelindent=5pt,style=multiline,leftmargin=3cm]
\item[permit-overrides]
                        Se la valutazione di una \epolicy restituisce \emph{permit}, allora il risultato è \emph{permit}.
                        In altre parole, \emph{permit} ha la precedenza, indipendente dal risultato delle altre \emph{policy}.
                        Invece, se c'è almeno una \epolicy che restituisce \emph{deny} e tutte le altre restituiscono
                        \texttt{not-app} o \emph{deny}, allora il risultato è \emph{deny}. Se tutte le policy restituiscono
                        \texttt{not-app}, allora il risultato è \texttt{not-app}. In tutti gli altri casi, il risultato è \texttt{indet}.
\end{description}
Se il risultato della decisione è \emph{permit} o \emph{deny}, ogni algoritmo restituisce una sequenza di \emph{fulfilled obbligations}
conforme alla strategia $\delta$ di \emph{fulfilment} scelta. Ci sono due possibili strategie:
\begin{description}[labelindent=5pt,style=multiline,leftmargin=3cm]
  \item[All]La strategia \emph{all} richiede la valutazione di tutte le \epolicy appartenenti alla sequenza in input e
            restituisce le \emph{fulfilled obligation} relative a tutte le decisioni.
  \item[Greedy]La strategia \emph{greedy} stabilisce che, se ad un certo punto, la valutazione della sequenza di \epolicy
               in input non può più cambiare, si può non esaminare le altre policy e terminare l'esecuzione. In questo modo
               si migliora le prestazioni della valutazione in quanto non si sprecano risorse computazionali
               per valutazioni di policy che non avrebbero alcun impatto sul risultato finale.
\end{description}\par
L'ultimo passo consiste nell'inviare la risposta del \ac{PDP} al \ac{PEP} per l'\emph{enforcement}. A questo scopo, il \ac{PEP} verifica
che tutti gli eventuali obblighi imposti dal \ac{PDP} al richiedente siano soddisfatti e decide, in base all'algoritmo di
enforcement scelto, il comportamento per le decisioni di tipo \texttt{not-app} e \texttt{indet}. Gli algoritmi sono:
\begin{description}[labelindent=5pt,style=multiline,leftmargin=4cm]
  \item[base]Il \ac{PEP} mantiene tutte le decisioni, ma se c'è un errore nella verifica degli obblighi il risultato è \texttt{indet}.
  \item[deny-biased]Il \ac{PEP} concede l'accesso solo nel caso in cui questa sia la decisione del \ac{PDP} e che gli eventuali obblighi imposti dal \ac{PDP} siano stati soddisfatti. In tutti gli altri casi, il \ac{PEP} nega l’accesso.
  \item[permit-biased]
  Questo algoritmo è il duale di deny-biased.
\end{description}
Questi algoritmi evidenziano il fatto che le \emph{obligations} non solo influenzano il processo di autorizzazione, ma anche l'enforcement.
Si nota infine che gli errori causati da obbligazioni opzionali sono ignorati.


\section{Esempio FACPL}
\label{sec:Esempio_FACPL}
In questo semplice esempio di politica in \ac{FACPL}, due utenti possono interagire con il sistema attraverso delle richieste.
John può scrivere sulla risorsa "file.txt", ma non vi è specificata nessuna regola per la lettura. Invece Tom può
leggere il "file.txt", ma non può scriverci.\par
\lstinputlisting[language = FACPL, caption = {Esempio di politica in FACPL}\label{lst:facpl_es_Pol}]{./code/cap3/esempioFacpl}
Nel PAS si indicano gli algoritmi usati e le richieste da includere.
In questo esempio entrambi gli utenti richiedono sia l'azione di "WRITE" che l'azione di "READ".
Qui di seguito si presenta la struttura delle \emph{request}.
\lstinputlisting[language = FACPL, caption = {Esempio di richieste in FACPL}\label{lst:facpl_es_Req}]{./code/cap3/esempioRichiesteFacpl}
L'output dopo l'esecuzione sarà il seguente:
\begin{verbatim}
  Request: Request1

    Authorization Decision: PERMIT
    Obligations: PERMIT M log_permit([John])

  Request: Request2

    Authorization Decision: NOT_APPLICABLE
    Obligations:

  Request: Request3

    Authorization Decision: PERMIT
    Obligations: PERMIT M log_permit([Tom])

  Request: Request4

  Authorization Decision: DENY
  Obligations: DENY M log_deny([Tom])
\end{verbatim}
La prima richiesta di scrittura da parte di John viene chiaramente accettata, la sua richiesta di lettura però ha
come risultato \emph{NOT APPLICABLE} in quanto non ci sono regole che possono essere usate per dare una valida risposta.
La richiesta di lettuta di Tom invece viene accettata, mentre la sua ultima richiesta di scrittura riceve un deny
perché è bloccata alla \emph{rule} writeRuleT.\par
Da questo esempio si può vedere la semplicità con cui si possono scrivere le policy e le richieste.
Le stesse politiche scritte in \ac{XML} oltre a risultare prolisse a confronto, sono difficili da comprendere o analizzare.\par
Infine si fa notare che in questa versione, le richieste in ingresso sono completamente indipendenti tra loro e
la parte dello Usage Control descritta in \ref{sec:Usage_Control}, in cui si enfatizza l'uso di controlli continuativi e
la gestione di contesti mutabili, non è ancora implementata. Nel capitolo successivo si mostra la nuova struttura per
per uno sviluppo basato proprio sullo Usage Control.

\chapter{Implementazione Usage Control in FACPL}
\label{chap:Implementazione Usage Control in FACPL}
\section{Il processo di valutazione}
\label{sec:Il processo di valutazione}
\section{Estensione Linguistica}
\label{sec:Estensione Linguistica}
\section{Semantica}
\label{sec:Semantica}
\section{Esempi}
\label{sec:Esempi}

\chapter{Esempi}
\label{chap:Esempi}

\section{Contatore}
\label{sec:Contatore}

\section{Data}
\label{sec:Data}

\section{Lettura e scrittura}
\label{sec:Lettura e scrittura}

\chapter{Strumenti usati per lo sviluppo}
\label{chap:Strumenti usati per lo sviluppo}

\section{XTEXT}
\label{sec:XTEXT}

\section{Plugin Eclipse}
\label{sec:Plugin Eclipse}

\myChapter{Conclusioni}
\label{chap:Conclusioni}
In questa tesi è stata sviluppata un'estensione del linguaggio \ac{FACPL} per migliorare la gestione delle richieste di accesso.
Nella prima parte sono state descritte le strutture su cui si è basato lo sviluppo. Sono stati introdotti i primi modelli
dell'Access Control partendo dal più semplice \ac{ACL} arrivando fino a \ac{PBAC}. In seguito è stato mostrato lo Usage Control
descrivendo il nuovo modello UCON\textsubscript{ABC} creato da Jaehong Park e Ravi Sandhu.

Successivamente è stato esaminato il linguaggio \ac{FACPL} su cui si è basato implementazione del monitor a runtime.
\MakeUppercase{è} stata mostrata la sintassi, la semantica e il processo di valutazione per presentare i lati positivi
e soprattutto i limiti del linguaggio su cui si è lavorato per migliorare il controllo continuativo degli accessi.

Sono state aggiunte nuove produzioni alla sintassi ed è stata modificata la risposta dell'autorizzazione per poter
essere utilizzata nel nuovo processo valutativo, in cui si elimina i controlli superflui per un determinato tipo di richieste.
Alle modifiche apportate sulla sintassi sono seguite le implementazioni di nuove strutture nella libreria Java su cui si
basa \ac{FACPL}. Infine sono stati proposti esempi per presentare le operazioni possibili e i miglioramenti nell'uso delle
risorse computative.

Il lavoro di estensione della libreria è stato svolto insieme al mio collega Federico Schipani. Nella tesi del
mio collega si parla di come è stato possibile esprimere in FACPL politiche di controllo degli accessi basate
sul comportamento passato. Le modifiche riguardano principalmente il \ac{PDP} (descritto nel Capitolo~\ref{chap:Formal Access Control Policy Language})
e consistono nella creazione di uno stato del sistema. In questa tesi invece si modifica il \ac{PEP}
(descritto ugualmente nel Capitolo~\ref{chap:Formal Access Control Policy Language}
e nel Capitolo~\ref{chap:Usage Control in FACPL} e \ref{chap:Estensione della libreria FACPL} nella versione modificata)
e si implementa un nuovo tipo di obbligazione.
\section{Sviluppi Futuri}
\label{sec:Sviluppi Futuri}
In questo documento sono stati esposti due esempi, in uno si usa il numero di richieste, nell'altro il tempo.
La libreria di \ac{FACPL} è semplice da estendere ed è facile pensare a nuovi possibili controlli su cui
le autorizzazioni possono basarsi. Si potrebbe associare le verifiche della richiesta alla posizione
geografica dell'utente, oppure alla congestione della rete aggiungendo semplicemente la gestione di un nuovo parametro nella
creazione delle obbligazioni e le rispettive funzioni che ne controllano il valore.

\begin{thebibliography}{99}

\bibitem{ACL}{
 J. Barkley (1997)-
 \emph{Comparing simple role based access control models and access control lists},
 In \emph{Proceedings of the second ACM workshop on Role-based access control}: 127-132,
 http://merlot.usc.edu/cs530-s04/papers/Barkley97a.pdf
}

\bibitem{RBAC}{
Sandhu, R., Ferraiolo, D.F. and Kuhn, D.R. (July 2000) -
\emph{The NIST Model for Role-Based Access Control: Toward a Unified Standard}: 47–63,
5th ACM Workshop Role-Based Access Control
http://csrc.nist.gov/rbac/sandhu-ferraiolo-kuhn-00.pdf
}

\bibitem{ABAC}{
Vincent C. Hu, David Ferraiolo, Rick Kuhn, Adam Schnitzer, Kenneth Sandlin, Robert Miller, Karen Scarfone (2014) -
\emph{SP 800-162, Guide to Attribute Based Access Control (ABAC) Definition and Considerations}: 6-13,
http://nvlpubs.nist.gov/nistpubs/SpecialPublications/NIST.SP.800-162.pdf
}

\bibitem{NISTACM}{NIST (2009) -
\emph{A survey of access Control Models}: 7-10,
\url{http://csrc.nist.gov/news_events/privilege-management-workshop/PvM-Model-Survey-Aug26-2009.pdf}
}

\bibitem{uconSN}{
Jaehong Park, Ravi Sandhu (2010) -
\emph{ A Position Paper: A Usage Control (UCON) Model for Social Networks Privacy}
}

\bibitem{ucon}{
Jaehong Park, Ravi Sandhu (2004) -
\emph{The UCON Usage Control Model}: 128-174,
\url{http://drjae.com/Publications_files/ucon-abc.pdf},
ACM Trans. Inf. Syst. Secur.
}

\bibitem{UCRT}{
Leanid Krautsevich, Aliaksandr Lazouski, Fabio Martinelli, Paolo Mori, Artsiom Yautsiukhin (2010) -
\emph{Usage Control, Risk and Trust},
7th International Conference on Trust, Privacy \& Security in Digital Busines (TRUSTBUS'10), Bilbao, Spain
}

\bibitem{IEEE}{
Alexander Pretschner, Manuel Hilty, Florian Schutz, Christian Schaefer, Thomas Wlater (2008) -
\emph{Usage Control Enforcement}: 44-53,
Published by the ieee Computer society,
\url{https://www.researchgate.net/profile/Thomas_Walter11/publication/3438132_Usage_Control_Enforcement_Present_and_Future/links/53f5caf20cf2888a7491e8d7.pdf}
}

\bibitem{SurveyUsageControl}{
Aliaksandr Lazouski, Fabio Martinelli, Paolo Mori (2010) -
\emph{Usage control in computer security: A Survey}: 81-99,
Computer Science Review
}

\bibitem{UsageControlCloud}{
Aliaksandr Lazouski, Gaetano Mancini, Fabio Martinelli, Paolo Mori (2012) -
\emph{Usage Control in Cloud Systems}: 202-207 -
Istituto di informatica e Telematica, Consiglio Nazionale delle Ricerche.
\url{http://ieeexplore.ieee.org/xpl/freeabs_all.jsp?arnumber=6470943}
}

\bibitem{fullfacpl}{
Andrea Margheri, Massimiliano Masi, Rosario Pugliese, Francesco Tiezzi -
\emph{A Formal Framework for Specification, Analysis and Enforcement of Access Control Policies}}

\bibitem{FacplGuide}{
FACPL guide -
\url{http://facpl.sourceforge.net/guide/facpl_guide.html}
}

\bibitem{xtend}{
Xtend Documentation -
\url{http://www.eclipse.org/xtend/documentation/index.html}
}

\bibitem{xtext}{
Xtext Documentation -
\url{http://www.eclipse.org/Xtext/documentation/index.html}
}

\end{thebibliography}

%--------------------------------------------------------------
\end{document}
%--------------------------------------------------------------
