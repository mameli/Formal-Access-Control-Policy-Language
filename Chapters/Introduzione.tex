\myChapter{Introduzione}
I sistemi informatici, fin dalla loro nascita, sono stati utilizzati per la gestione di dati.
Il tipo di informazioni gestite ha reso necessario la creazione di un sistema che le proteggesse, infatti i
dati più sensibili, se diffusi senza una valida autorizzazione, possono arrecare ad esempio danni economici
ad una società o nuocere anche gli utenti nel privato.
Per far fronte al problema sono stati sviluppati dei modelli per il controllo degli accessi. Tuttavia la quantità di dati
e la complessità dei sistemi moderni ha mostrato i limiti delle tecnologie concepite e questo fatto ha portato inevitabilmente
allo sviluppo di un nuovo approccio: lo \ac{UCON}.

I ricercatori dell'università di San Antonio, Jaehong Park e Ravi Sandhu, hanno descritto un modello
di \ac{UCON}\cite{ucon} che consente di ottenere decisioni durante l'accesso e di basare le valutazioni di una richiesta
sul comportamento passato.

In questa tesi si descrive un'estensione basata su UCON per \ac{FACPL}.
Il problema principale da risolvere è la staticità del processo di valutazione del linguaggio.
Tutte le richieste di accesso ricevono una risposta attraverso un sistema di verifica sempre uguale e i controlli
risultano ridondati e in alcuni casi inutili.
Nella prima parte dello sviluppo è stata modificata la sintassi del linguaggio per adeguarla ad una nuova
gestione delle richieste, in seguito
è stata sviluppata un'estensione della libreria su cui si basa \ac{FACPL} per modificare
il processo valutativo al seguito di richieste di accesso continuative.

Il lavoro di estensione della libreria è stato svolto insieme al mio collega Federico Schipani. Nella tesi del
mio collega si parla di come è stato possibile esprimere in FACPL politiche di controllo degli accessi basate
sul comportamento passato. Le modifiche riguardano principalmente il \ac{PDP} (descritto nel Capitolo~\ref{chap:Formal Access Control Policy Language})
e consistono nella creazione di uno stato del sistema. In questa tesi invece si modifica il \ac{PEP}
(descritto ugualmente nel Capitolo~\ref{chap:Formal Access Control Policy Language}
e nel Capitolo~\ref{chap:Usage Control in FACPL} e \ref{chap:Estensione della libreria FACPL} nella versione modificata)
e si aggiunge un nuovo tipo di obbligazione.
Nei capitoli successivi si mostrano i passi che hanno portato allo sviluppo della nuova gestione delle richieste
in grado di rendere dinamico il processo di valutazione.
\vspace{5mm}
\newline
Il resto del documento è così strutturato:
\begin{itemize}
  \renewcommand\labelitemi{--}
  \item Nel Capitolo~\ref{chap:Access Control e Usage Control}
  si introducono vari modelli di Access Control, si descrive lo \ac{UCON} e si mostrano due casi di studio basati sul nuovo modello.
  \item Nel Capitolo~\ref{chap:Formal Access Control Policy Language}
  si descrive la sintassi e la semantica di \ac{FACPL}, il processo di valutazione e un esempio utilizzando il linguaggio.
  \item Nel Capitolo~\ref{chap:Usage Control in FACPL}
  si mostra l'estensione di \ac{FACPL} per migliorare la gestione delle richieste.
  \item Nel Capitolo~\ref{chap:Estensione della libreria FACPL}
  si riportano le modifiche alla libreria Java su cui è basato il linguaggio per adattarlo al nuovo processo valutativo.
  \item Nel Capitolo~\ref{chap:Conclusioni}
  si riassume il lavoro svolto e si presentano alcuni possibili sviluppi futuri.
\end{itemize}
