\documentclass[10pt]{article}

\usepackage[italian]{babel}
\usepackage[utf8]{inputenc}
\usepackage{hyperref}
\usepackage[normalem]{ulem}
\usepackage[font=large, tablename= ]{caption}
\usepackage{fixltx2e}

\useunder{\uline}{\ul}{}
\renewcommand{\baselinestretch}{1.3}
\setlength{\textwidth}{16 cm}
\setlength{\oddsidemargin}{0 cm}
\setlength{\topmargin}{-1.5 cm}
\setlength{\textheight}{30 cm}
\begin{document}

\begin{table}[]
\centering
\caption*{Progetto e implementazione in FACPL di un monitor a runtime per il supporto al controllo continuativo degli accessi}
\begin{tabular}{lll}
\textbf{Candidato:}   & Filippo Mameli & \href{mailto:filippo.mameli@stud.unifi.it}{\texttt{filippo.mameli@stud.unifi.it}}  \\
\textbf{Relatore:}    & Rosario Pugliese  & \href{mailto:rosario.pugliese@unifi.it}{\texttt{rosario.pugliese@unifi.it}}                           \\
\textbf{Correlatore:} & Andrea Margheri   & \href{mailto:andrea.margheri@unifi.it}{\texttt{andrea.margheri@unifi.it}}
\end{tabular}
\end{table}
\subsection*{Riassunto}
La tesi propone lo sviluppo di un'estensione, basata su UCON\textsubscript{ABC}, del linguaggio
Formal Access Control Policy Language (FACPL)
in modo tale da migliorare la gestione delle richieste di accesso a runtime.

Nella prima parte sono state descritte le strutture su cui si è basato lo sviluppo. Sono stati introdotti alcuni
modelli di Access Control, il modello UCON\textsubscript{ABC} e si è analizzato il linguaggio FACPL.

FACPL è stato creato come alternativa al linguaggio
eXtensible Access Control Markup Language (XACML) per sviluppare sistemi basati sul modello per il controllo degli
accessi Policy Based Access Control (PBAC). Questo è supportato da una libreria Java e ha l'obiettivo di rendere semplice
la scrittura di politiche e di definire un framework costruito sopra basi formali solide.

L'obiettivo della tesi è creare un'estensione in grado di modificare il processo valutativo di FACPL.
Lo sviluppo è stato diviso in due parti. In primis la struttura del linguaggio è stata modificata di modo
che si possa creare delle politiche che si basino sul comportamento passato. In seguito il linguaggio
è stato ulteriormente esteso per migliorare la gestione dell'utilizzo continuativo di risorse.
In questa tesi si espone il secondo sviluppo, per la prima parte dell'estensione si rimanda
alla tesi del mio collega Federico Schipani, con cui ho lavorato per l'implementazione dell'intera nuova struttura.

Il problema principale da risolvere è la staticità del processo di valutazione del linguaggio.
Tutte le richieste di accesso ricevono una risposta attraverso un sistema di verifica che esegue sempre
tutti i controlli. Non modificando mai i passi della valutazione, vari controlli per le richieste potrebbero
risultare ridondati e in alcuni casi inutili. Le richieste che non modificano i dati
o lo stato del sistema, come ad esempio le letture su file, passano per lo stesso processo di valutazione
di tutte le altre richieste e il sistema dei controlli risulta eccessivamente complesso per l'autorizzazione di azioni semplici.
Per questo motivo è stato ideato un monitor in grado di scegliere a runtime il processo di verifica delle richieste.\par

La descrizione dell'estensione continua mostrando le modifiche e le strutture aggiunte alla libreria Java che supporta
FACPL. \MakeUppercase{è} stato mostrato il plugin Eclipse con cui si può trasformare un file scritto in FACPL nei
relativi file Java e sono stati presentati due esempi utilizzando la nuova estensione. Nell'ultima parte
sono stati mostrati i miglioramenti nell'uso delle risorse computative attraverso una serie di benchmark.

\end{document}
