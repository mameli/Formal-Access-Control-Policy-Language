\myChapter{Usage Control in FACPL}
\label{chap:Usage Control in FACPL}
In questo capitolo si descrive l'estensione del linguaggio FACPL per implementare lo Usage Control.
Più precisamente lo sviluppo è stato diviso in due parti. In primis la struttura è stata modificata in modo
in cui si potesse creare delle politiche che si basassero sul comportamento passato. In seguito il linguaggio
è stato ulteriormente esteso per migliorare la gestione dell'utilizzo continuativo di risorse .\par
In questo capitolo e nel succesivo si espone il secondo sviluppo, per la prima parte dell'estensione si rimanda
alla tesi del mio collega Federico Schipani, con cui ho lavorato per l'implementazione dell'intera nuova struttura.\\ \par
Il linguaggio è stato esteso in modo tale da ottimizzare la gestione di un insieme di richieste sulla stessa risorsa.
Per far questo è stato necessaria una modifica sul processo di valutazione e l'inserimento di nuovi componenti.
\section{Il processo di valutazione}
\label{sec:Il processo di valutazione}
Per memorizzare il comportamento passato, è stato essenziale l'aggiunta di un altro elemento nel processo di valutazione.
Come si vede in figura \ref{fig:img/evaluationProcessStatus}, la nuova componente è Status.\par
\MyFigure{img/evaluationProcessStatus}{Processo di valutazione Status}{1}
Oltre agli \emph{attribute names} il Context Handler avrà anche la possibilità di ricavare gli \emph{status attributes}.
Il PDP quindi potrà richiedere entrambi i tipi di attributi per elaborare la sua risposta. La decisione poi passerà al PEP
come avveniva precedentemente.\par
Inoltre sono state aggiunte un tipo di obbligazioni che possono modificare lo Status. Il PEP quindi può far eseguire questo tipo di
azioni per cambiare gli attributi.\\\par
Riprendendo il primo esempio \ref{subs:Accessi in lettura e scrittura di file}, si riporta \emph{Se uno degli amministratori
sta scrivendo su un file, nessuno può leggerlo o apportare modifiche nello stesso momento}. Questa regola è realizzabile
solo con l'utilizzo di uno status attribute(e.g.\ un booleano isWriting) che viene modificato da un'azione del PEP dopo che
una richiesta di scrittura è stata accettata. Il PDP nelle successive richieste di lettura o di scritturà farà un controllo
sull'attributo e negherà di conseguenza l'accesso.\\\par
\MyFigure{img/evaluationProcessPEP}{Processo di valutazione PEP-Check}{1}
Il passo successivo consiste nell'eliminazione della ridondanza dei controlli da parte del PDP.\par
Quando in \ref{subs:Accessi in lettura e scrittura di file}, si scrive:
\emph{Letture ripetute, su uno stesso insieme di file, vengono gestite in modo efficiente};
Si fa riferimento al nuovo metodo per il controllo di richieste uniformi su una categoria di risorse e
in figura \ref{fig:img/evaluationProcessPEP} si mostra il processo di valutazione ridotto.\\\par
Si può definire un certo tipo di richieste che verrà interamente gestito dal PEP dopo una prima, e unica,
valutazione del PDP. Se un sistema deve gestire numerose richieste di lettura, e non ci sono elementi che bloccano
un accesso, è possibile accettare una richiesta esaminando soltanto la tipologia e l'appartenenza di una risorsa ad
un insieme predefinito.\\\par
Prendiamo per esempio un sistema che gestisce dei file dentro delle cartelle. Supponiamo che ci sia una cartella
"Share" dove tutti gli utenti possono leggere solo se non ci sono utenti che scrivono.
Il sistema può gestire richieste di lettura ripetute, facendo un unico controllo per la verifica dell'assenza di
utenti che stanno scrivendo, alla prima domanda di accesso. Poi le richieste successive potranno essere
gestite dal PEP con solo due controlli, uno sul tipo di azione, in questo caso una lettura, l'altro sull'appartenza
alla cartella "Share". Se i due controlli non sono validi si ritorna alla valutazione completa con il PDP.
\section{Sintassi e semantica}
\label{sec:Sintassi e semantica}
La modifica nella struttura ha reso necessario apportare dei cambiamenti anche nella grammatica.\par
Al PAS è stato aggiunto \status della forma:
$$(status: Attribute^+)^?$$
quindi lo \status è un elemento opzionale e può essere formato da almeno un \emph{ Attribute}.\\\par
\emph{ Attribute} invece è della forma:
$$(Type\ Identifier (= Value)^?)$$
quindi è formato da un tipo, che può essere \emph{int}, \emph{float}, \emph{boolean} o \emph{date}, da una stringa
identificatrice e da un valore.\par
Le \emph{PepAction} sono state modificate in modo tale da eseguire operazioni sugli attributi dello stato
e sono della forma: \emph{nomeAzione}(\emph{Attribute},type)
per esempio l'azione di somma di interi è \emph{add}(\emph{Attribute},int).\par
Agli \emph{Attribute Names} è stata aggiunta la produzione \textit{Status/Identifier} per identificare i
termini dello Status nelle \emph{Expressions}.\par
La nuova produzione per le \emph{Obligations} della forma:
$$[Effect\ ObType\ (Expr)^*\ CheckValue^?]$$
è utilizzata per la nuova gestione delle richieste da parte del PEP. La nuova tipologia
non ha pepAction, in quanto il PEP può modificare valori dello \status solo se il processo di valutazione è quello
completo, cioè quello che include anche i controlli eseguiti dal PDP. Inoltre deve esserci un insieme di \emph{Expression}
affinchè il PEP possa fare le proprie verifiche e può esserci un valore opzionale che indica il tipo della scadenza.
Da ora in poi il nuovo tipo di obbligazione sarà denominato \emph{Obligation Check.}\\
Infine i \emph{Check Values} della forma:
$$cValue ::= Int\ |\ StringDate$$
Indicheranno la scadenza delle \emph{Obligation Check}. Int determina il numero massimo di richieste, mentre StringDate
il limite temporale. Se CheckType viene omesso la \eobligation non ha scadenza.\par
\begin{table}[h]
\centering
\scriptsize
\caption{Sintassi di $FACPL_{PB}$}
\hrule
$
\begin{array}{@{\,}r@{\ \ }r@{\ }r@{\ \ }l@{\ }}

&&&\\[-.2cm]
{\textbf{Policy Authorisation Systems}} &
\mathit{PAS} & ::= & ( \,  \x{pep:} \, \mathit{EnfAlg}\ \ \x{pdp:}\, \mathit{PDP} \, \  \x{status:}\, \mathit{
[Attribute]})
\\[.2cm]
{\textbf{Attribute}} &
\mathit{Attribute}
& ::= & (\mathit{Type} \ \mathit{Identifier} \ (= Value )^{?})
\\[.2cm]
{\textbf{Type}} &
\mathit{Type}
& ::= & \x{int} \ | \ \x{boolean} \ | \ \x{date} \ | \ \x{double}
\\[.2cm]
{\textbf{Enforcement algorithms}} &
\mathit{EnfAlg}
& ::= & \based \Sep \denyBiased \Sep \permitBiased
\\[.2cm]
{\textbf{Policy Decision Points}} &
\mathit{PDP} & ::= & \pdpPol{\algNT\ }{\x{policies:} \, \mathit{Policy}^{+}}
\\[.2cm]
{\textbf{Combining algorithms}} &
\algNT & ::= & \permitOver \Sep \denyOver \Sep \denyUnless \Sep \permitUnless \\
&& \mid &
\firstApp \Sep \onlyOneApp \Sep \weakCon \Sep \strongCon
\\[.2cm]
{\textbf{Policies}} &
\mathit{Policy} & ::= &
\ruleOpt{\mathit{Effect}\ \ \x{target:} \, Expr\ \ \x{obl:} \, \mathit{Obligation}^{*} \, } \\
&& \mid &
\{ \algNT\ \ \x{target:} \, Expr\ \
\x{policies:} \, \mathit{Policy}^{+}  \ \ \x{obl:} \, \mathit{Obligation}^{*} \, \}
\\[.2cm]
{\textbf{Effects}} &
\mathit{Effect} & ::= & \permit \Sep \deny
\\[.2cm]
{\textbf{Obligations}} &
\mathit{Obligation} & ::= & [ \, \mathit{Effect} \ \ \mathit{ObType} \ \ \obl{Expr} \, ]\\
&& & \Sep [ \, \mathit{Effect} \ \ \x{env:} \, \mathit{Expr}\ \  \x{status:}\mathit{Expr} \ \ \x{exp:}\mathit{Value}^?\, ]
\\[.2cm]
{\textbf{PepAction}} & \mathit{PepAction} & ::= & \, \x{add(\mathit{Attribute}, int)} \Sep \x{flag(\mathit{Attribute}, boolean)} \\
&& & \Sep \x{sumDate(\mathit{Attribute}, date)} \Sep \x{div(\mathit{Attribute}, int)} \\
&& & \Sep \x{add(\mathit{Attribute}, float)} \ \Sep \x{mul(\mathit{Attribute}, float)} \\
&& & \Sep \x{mul(\mathit{Attribute}, int)} \ \Sep \x{div(\mathit{Attribute}, float)} \\
&& & \Sep \x{sub(\mathit{Attribute}, int)} \ \Sep \x{sub(\mathit{Attribute}, float)} \\
&& & \Sep \x{sumString(\mathit{Attribute}, string)} \\
&& & \Sep \x{setValue(\mathit{Attribute}, string)}\\
&& & \Sep \x{setDate(\mathit{Attribute}, date)}
\\[.2cm]
{\textbf{Obligation Types}} &
\mathit{ObType} & ::= & M \Sep O
\\[.4cm]
\textbf{Expressions}&
\mathit{Expr} & ::= &
\mathit{Name} \Sep \mathit{Value}  \\
& & \mid &\x{and(\mathit{Expr}, \mathit{Expr})} \Sep \x{or(\mathit{Expr}, \mathit{Expr})} \Sep \x{not(\mathit{Expr})} \\
& & \mid &
 \x{equal(\mathit{Expr},\mathit{Expr})}  \Sep \x{in}(\mathit{Expr}, \mathit{Expr}) \\
& & \mid & \x{greater}\textrm{-}\x{than(\mathit{Expr},\mathit{Expr})} \Sep \x{add(\mathit{Expr} ,\mathit{Expr} )}\\
& & \mid & \x{subtract(\mathit{Expr} ,\mathit{Expr} )} \Sep \x{divide(\mathit{Expr} ,\mathit{Expr} )}\\
& & \mid & \x{multiply(\mathit{Expr} ,\mathit{Expr} )}  \Sep \x{less}\textrm{-}\x{than(\mathit{Expr}, \mathit{Expr})}\\
\\[.2cm]
%
\textbf{Attribute Names} &
\mathit{Name} & ::= & \mathit{Identifier}/\mathit{Identifier} \ | \ \x{status}/\mathit{Identifier}\\[.2cm]
%
\textbf{Literal Values} &
\mathit{Value} & ::= & \x{true} \mid \x{false} \mid \mathit{Double} \mid \mathit{String} \mid \mathit{Date}
\\[.4cm]
{\textbf{Requests}} &
\mathit{Request} & ::= & {\attribute{\mathit{Name}}{\mathit{Value}}}^{+}
\\[.1cm]
\end{array}
$
\hrule
\label{tab:facpl_new_syntax}
\end{table}

Si propone adesso un esempio di una Policy che utilizza il nuovo tipo di obbligazione.\par
Nella Policy si vuole specificare che un utente di nome \emph{Charlie} può effetture una lettura solo se nessuno sta scrivendo
e vuole utilizzare la risorsa \emph{contabilita.xlsx}.
Se la richiesta viene accettata si deve cambiare l' attributo \emph{isReading} a true, oltre a ciò al susseguirsi
di un altra richiesta di lettura il sistema cambiarà il tipo di processo valutativo e il PEP effettuerà i controlli
solo sul tipo di azione e sul nome della risorsa.\par
\lstinputlisting[language = FACPL, caption = {Esempio per la sintassi}\label{lst:esempio_sintassi}]{code/cap4/charlieRead}
\par Si fa notare che con questa porzione di codice:
\lstinputlisting[firstline = 8, lastline = 8,language = FACPL, caption = {Esempio per la sintassi}\label{lst:esempio_sintassi}]{code/cap4/charlieRead}
si esprime la nuova struttura per la creazione delle \emph{Obligation Check} ed è questo il costrutto
in cui si indicano i controlli che il PEP deve effettuare.\\\par
\subsection*{Semantica}
\label{sub:Semantica_Ext}
La trasformazione della sintassi ha determinato anche variazioni nella semantica descritta in \ref{sec:sem_fpl}.\par
Il PDP fa uso degli \emph{Status Attributes} per la valutazione di una richiesta in input.
Questi attributi sono modificabili tramite alcune azioni dette \emph{Pep Action}, come l'operazione di somma \emph{add(Attribute,int)}
oppure l'assegnamento di una data con \emph{setDate(Attribute,date)}.
Sono definiti nel PAS come descritto nell'esempio precedente con questa struttura:
\lstinputlisting[firstline = 18, lastline = 18,language = FACPL, caption = {Esempio per la sintassi}\label{lst:esempio_sintassi}]{code/cap4/charlieRead}
\par Il PEP oltre a verificare che le Pep Action siano eseguite, con l'aggiunta delle \emph{Obbligation Check}, ha il
controllo nella gestione delle richieste continuative. Se nella policy è presente una obbligazione che fa parte del nuovo tipo,
il PEP ha il compito di controllare che le \emph{Expression} contenute nell'\emph{Obligation Check} siano vere.\par
Nell'esempio precedente l'obbligazione è definita con:
\lstinputlisting[firstline = 8, lastline = 8,language = FACPL, caption = {Esempio per la sintassi}\label{lst:esempio_sintassi}]{code/cap4/charlieRead}
il PEP deve verificare quindi che l'azione richiesta sia una "read" e che la risorsia sia "contabilita.xlsx".
Se il controllo dà esito positivo la richiesta è subito accettata, se la richiesta non passa una delle verifiche si continuerà
il processo di valutazione ripartendo dal PDP e svolgendo tutti i passi.\\\par
Le \emph{Obligation Check} possono essere permanenti, limitate temporalmente oppure limitate sul numero di richieste
accettate.\par
\begin{description}
  \item[Permanenti]Se sono del primo tipo e la verifica sulle \emph{Expression} è sempre vera, il PEP continuerà a gestire le
  richieste senza il controllo del PDP. La gestione cambia solo quando una richiesta non è conforme a quelle accettate.\par
  \item[Scadenza sul tempo]Se sono del secondo tipo, anche se la verifica sulle \emph{Expression} è vera, quando il tempo limite è stato superato,
  la gestione delle richieste cambia e si riprende il processo valutativo completo.
  \item[Scadenza sul numero di richieste]Il terzo tipo di \emph{Obligation Check} è simile al secondo, con l'unica differenza che il limite non è temporale,
  ma sul numero di richieste elaborate.
\end{description}

\section{Esempi}
\label{sec:Esempi}
In questa sessione si mostrano gli esempio descritti in \ref{subs:es_UC} utilizzando le nuove funzionalità.
\subsection{Accessi in lettura e scrittura di file}
\label{sub:RW_Code_Sec}
\lstinputlisting[language = FACPL, caption = {Policy per esempio lettura e scrittura }\label{lst:esempio_RWP}]{code/cap4/checkReadWritePolicy}
\lstinputlisting[firstline = 1, lastline = 35, language = FACPL, caption = {Richieste per esempio lettura e scrittura }\label{lst:esempio_RWR}]{code/cap4/checkReadWriteRequests}
\subsection{Servizio di streaming}
\label{sub:Stream_Code_Sec}
\lstinputlisting[language = FACPL, caption = {Policy per esempio servizio streaming }\label{lst:esempio_SP}]{code/cap4/checkStreamingPolicy}
\lstinputlisting[language = FACPL, caption = {Richieste per esempio servizio streaming }\label{lst:esempio_SR}]{code/cap4/checkStreamingRequests}
