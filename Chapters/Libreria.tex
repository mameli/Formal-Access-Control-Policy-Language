\myChapter{Estensione della libreria FACPL}
\label{chap:Estensione della libreria FACPL}
La librearia \ac{FACPL} è basata su Java. Per implementare le nuove funzioni sono state aggiunte varie classi e sono state modificate
alcune parti per adattare il comportamento ai nuovi componenti e alle nuove strutture.\par
Nelle sezioni successive si descriveranno gli aspetti più rilevanti del codice. L'intera libreria si può trovare su
GitHub all'indirizzo
\begin{center}
  \url{https://github.com/andreamargheri/FACPL}
\end{center}
Lo sviluppo dell'estensione si può dividere in due passi principali.
La prima parte~(Serzione \ref{sec:PEPCheck}) consiste nel creare una classe \ac{PEP} che può gestire le richieste e adattare il processo di valutazione.
La seconda parte~(Sezioni \ref{sec:Le Obligations Check} e \ref{sec:FulfilledObligationCheck})
invece comprende l'implementazione di una gerarchia di \emph{Obbligations} che possono essere
sfruttate dal \ac{PEP} per la nuova gestione delle \emph{request}. \par
Nella Sezione~(\ref{sec:Esempi in Java}) si descrivono le classi java associate ai file \ac{FACPL}
dell'esempio in \ref{sub:Stream_Code_Sec}, inoltre si mostrano i miglioramenti
prestazionali utilizzando l'esempio sviluppato in \ref{sub:RW_Code_Sec}.
\section{La classe PEPCheck}
\label{sec:PEPCheck}
Il normale flusso del processo valutativo passa prima dal \ac{PEP}. Questo riceve la richiesta e poi la manda al
PDP affinchè possa valutarla. Infine la risposta del PDP è passata al \ac{PEP} che controlla le obbligazioni e
elabora la decisione conclusiva.
Il processo di valutazione ridotto invece avviene interamente nel \ac{PEP} che usa solo il \emph{Context Handler}
per ricavare gli attributi dell'ambiente e dello stato.

Avendo bisogno di entrambe le modalità di valutazione, una per avere la decisione completa e l'altra per eliminare la
ridondanza di verifiche da parte del \ac{PDP} su richieste ripetute,
la soluzione più appropriata è stata quella di creare un monitor
in cui si potesse valutare l'esigenza d'uso di uno o dell'alto flusso valutativo.

La classe PEPCheck è il monitor che esegue i controlli per determinare la gestione più adatta al contesto.
Questa estende la classe \ac{PEP} e contiente come campo privato il \ac{PDP} e la sua autorizzazione.
\javaFirstline{PEPCheck.java}{Costruttore}{22}{34}{PEPCostruttore}
L'elemento che contraddistingue di più la nuova classe dal precedente \ac{PEP} è la lista di
\emph{AbstractFulfilledObligationCheck} che è usata nel metodo \emph{doAuthorisation}
per prendere la decisione sul flusso valutativo. I casi da controllare sono tre:
\begin{description}

  \item[\normalfont\bfseries{\MakeUppercase{N}on ci sono \emph{Obligation Check} in lista}]
  {
  \hfill \\
  Non si apporta nessun cambiamento al normale flusso del processo valutativo.
  Si invia la richiesta al PDP che poi la passa la sua decisione al PEP.
  Il codice successivo mostra la condizione che rende possibile il paradigma.
  \javaLastline{PEPCheck.java}{Processo senza obligation check}{51}{57}{PEPNoCheck}
  }

  \item[\normalfont\bfseries{\MakeUppercase{N}ella lista c'è almeno una \emph{Obligation Check}}]
  {
  \hfill \\
  Il sistema valuta le \emph{Expression} associate alle obbligazioni con il metodo
  \emph{doPEPCheck}(vedi \ref{sec:Il processo di valutazione Facpl}
  per la struttura). Nel caso in cui la decisione sia \texttt{PERMIT} oppure \texttt{DENY},
  il risultato viene subito restituito senza i controlli del PDP.
  \javaLastline{PEPCheck.java}{Processo con obligation check}{59}{67}{PEPCheck}
  }

  \item[\normalfont\bfseries{\MakeUppercase{I}l controllo sulla lista restituisce \texttt{error/\MakeUppercase{NOT\_APPLICABLE}}}]
  {
  \hfill \\
  Se la decisione non ricade nei casi precedenti,
  si è verificato un errore nella valutazione delle espressioni. Se ci sono problemi nei controlli, il PEPCheck, prima di
  inviare la risposta, esegue una valutazione completa facendo \emph{doAuthorisation} sul \ac{PDP} e \emph{doEnforcement}.
  \javaFirstlineN{PEPCheck.java}{\texttt{error/NOT\_APPLICABLE} durante il processo}{66}{77}{PEPError}{28}
  }

\end{description}

Il \emph{doEnforcement} nel codice \ref{lst:PEPError} del PEPCheck richiama il metodo implementato in \ac{PEP}
nella linea \texttt{super.doEnforcement(authPDP, cxtReq);} come si mostra di seguito.
\javaLastline{PEPCheck.java}{Invocazione metodo doEnforcement su PEP}{82}{93}{PEPenforce}
Se la decisione non è \texttt{PERMIT} viene restituita subito la risposta negativa sull'autorizzazione e si conclude la valutazione.
Se invece la decisione è positiva si esegue un controllo sulle \emph{FulfilledObligations} e
nel caso in cui queste siano FulfilledObligationNumCheck oppure FulfilledObligationTimeCheck,
sono aggiunte alla lista di AbstractFulfilledObligationNumCheck di PEPCheck. Solo dopo l'inserimento delle obbligazioni
si restituisce la decisione ricavata.
\javaLastline{PEPCheck.java}{Aggiunta delle FulfilledObligationNumCheck}{95}{124}{PEPAddCheck}

La lista modificata in \ref{lst:PEPAddCheck} è utilizzata nel metodo \emph{doPEPCheck}(nel listato \ref{lst:PEPdoPEP})
per eseguire la valutazione di ogni singola obligation check senza passare dal \ac{PDP}.
Questo metodo controlla che i valori siano corretti confrontandoli con gli attributi del contesto. Se non ci
sono errori e le espressioni hanno superato i controlli, si ritiene la richiesta verificata
utilizzando il processo di valutazione ridotto.
\javaLastline{PEPCheck.java}{Valutazione FulfilledObligationNumCheck}{130}{153}{PEPdoPEP}

Nella figura \ref{fig:img/classDiagramPep.png} si mostra il grafico UML della classe PEPCheck e i relativi elementi
associati.
\MyF{img/classDiagramPep.png}{Diagramma PEPCheck}{1}{15cm}

Nella sezione successiva \ref{sec:Le Obligations Check} si descrivono le Obligations utilizzate
nel metodo \emph{doPEPCheck} listato in \ref{lst:PEPdoPEP}.

\section{Le Obligations Check}
\label{sec:Le Obligations Check}
La libreria è stata estesa con le \emph{Obbligations Check} perché c'era bisogno di obbligazioni che avessero solo
\emph{Expressions} da verificare senza avere nella struttura le pepAction.
La \emph{Obbligation Check} è una classe che estende la preesistente \emph{AbstractObligation}.
Nel codice \ref{lst:OBLConstr} si mostra che la classe ha come campi due alberi di espressioni booleane, che possono
a loro volta contenere altre \emph{expressions}, due classi per indicare la scadenza e una variabile che determina
il tipo di obligation check.
\javaFirstline{ObligationCheck.java}{Costruttore e campi}{12}{31}{OBLConstr}
La scadenza, come già detto in \ref{sub:Semantica_Ext}, può essere di tre tipi: persistente,
limitata temporalmente oppure limitata dal numero di richieste gestite. I tipi sono indicati con "P" per le obbligation Persistenti,
"T" per le Temporali e "N" per quelle Numeriche come si mostra nell'\texttt{enum} in \ref{lst:OBLType}.
\javaLastline{CheckObligationType.java}{Enum delle Obligations}{1}{5}{OBLType}
Per ogni obbligazione ci sono quattro costruttori in modo tale da poter creare l'oggetto per ogni combinazione di \emph{Expressions}
in input. In \ref{lst:OBLConstrT} si mostra un costruttore per le Obligation Temporali che prende in input due ExpressionBooleanTree,
la scadenza determinata da \emph{FacplDate expiration} e l'effetto (che può essere \texttt{PERMIT} oppure \texttt{DENY}).
\javaLastline{ObligationCheck.java}{Costruttore Obligation Temporale}{127}{133}{OBLConstrT}
Il metodo \emph{init} in \ref{lst:OBLInit}, presente in tutti i costruttori, inizializza i campi della classe a seconda del tipo di scadenza data
in input. Se la scadenza non è presenste si utilizza il metodo init per creare le obbligazioni di tipo Persistente.
\javaLastline{ObligationCheck.java}{init}{59}{74}{OBLInit}

In \emph{createObligation} in \ref{lst:OBLFul} si utilizzano i campi assegnati dai costruttori
e da init per la creazione delle \emph{FulfilledObligationNumCheck}
che sono descritte più approfonditamente in \ref{sec:FulfilledObligationCheck}.
\javaLastline{ObligationCheck.java}{Creazione di FulfilledObligationNumCheck}{153}{168}{OBLFul}

Nella figura \ref{fig:img/classDiagramObl.png} si mostra il diagramma della classe ObligationCheck.
\MyFigure{img/classDiagramObl.png}{Diagramma ObligationCheck}{1}
\section{Fulfilled Obligation Check}
\label{sec:FulfilledObligationCheck}
Le \emph{Fulfilled Obligation Check} sono una gerarchia di classi che è stata aggiunta
come estensione alla \emph{AbstractFulfilledObligation} già presente nella libreria.
Ogni tipo di Obligation check ha una Fulfilled Obligation Check associata. Quindi sono state create
tre tipi di Fulfilled:
\begin{enumerate}
  \item FulfilledObligationNumCheck
  \item FulfilledObligationCheckPersistent
  \item FulfilledObligationTimeCheck
\end{enumerate}
Le tre \emph{Fulfilled Obligation Check} sono molto simili
tra loro e l'unica differenza è la loro gestione della scadenza.\par
In figura \ref{fig:img/classDiagramFulfilled.png} si mostra il diagramma della gerarchia di classi.
\clearpage
\MyF{img/classDiagramFulfilled.png}{Diagramma FulfilledObligationCheck}{1.2}{22.5cm}
\clearpage
Quando si crea una Fulfilled Obligation con una scadenza determinata da un numero intero, si utilizza
la \emph{FulfilledObligationNumCheck}. In \ref{lst:FULn} si mostra il costruttore di questo tipo di
FulfilledObligation nel caso in cui la prima espressione sia una \emph{ExpressionFunction}, la seconda
una \emph{ExpressionBooleanTree} e la scadenza sia, ovviamente, un intero.
\javaFirstline{FulfilledObligationCheck.java}{FulfilledObligationNumCheck N}{58}{66}{FULn}

In \ref{lst:FULeV} e in \ref{lst:FULeS} si mostra il funzionamento di \emph{GetObligationResult}.
Questo metodo ha il compito, sia di verificare il valore di
verità delle espressioni associate, sia di gestire il valore di \emph{expiration}.
\begin{description}
  \item[\normalfont\bfseries{\MakeUppercase{E}xpiration valido}]
  {
  \hfill \\
  Se l'intero che identifica la scadenza è maggiore di zero,
  si possono valutare le \emph{Expressions} e richiamare il metodo \emph{subExpiration}(listato \ref{lst:FULnSub})
  per aggiornare il valore.
  \javaLastline{FulfilledObligationCheck.java}{Condizione di expiration valido}{94}{100}{FULeV}
  }
  \item[\normalfont\bfseries{\MakeUppercase{E}xpiration scaduta}]
  {
  \hfill \\
  Se si raggiunge la scadenza(in questo caso quando expiration è zero), le expressions non sono valutate
  e si assegna direttamente il valore \texttt{ERROR}.
  \javaLastline{FulfilledObligationCheck.java}{Condizione di expiration scaduta}{101}{105}{FULeS}
  }
\end{description}
In \ref{lst:FULeV} si richiama \emph{subExpiration}. Il metodo è utilizzato per aggiornare il
valore della scadenza. Come si vede nel listato \ref{lst:FULnSub} se il valore è -1 non si modifica expiration
(si spiegherà il motivo in seguito nel paragrafo delle FulfilledObligationCheckPersistent).
Negli altri casi si sottrae una unità a \emph{expiration} e
se si raggiunge il valore zero, si assegna al campo hasExpired \texttt{true}.
\javaLastline{FulfilledObligationCheck.java}{subExpiration}{147}{159}{FULnSub}

La Fulfilled persistente estende la classe FulfilledObligationNumCheck.
Il costruttore richiama la classe superiore e inizializza la scadenza numerica con
il valore speciale -1. Come si è mostrato in nel codice \ref{lst:FULnSub}, se \emph{expiration= -1}
il metodo non modifica mai il valore. In questo modo in \emph{GetObligationResult} di
FulfilledObligationNumCheck non si raggiungerà
il limite, le funzioni non restituiranno mai il risultato \texttt{ExpressionValue.ERROR} associato
ad un obbligazione scaduta e si valuteranno sempre le \emph{Expressions}.
Nel codice \ref{lst:FULp} si mostra il costruttore per FulfilledObligationCheck di tipo "P".
\javaFirstline{FulfilledObligationCheckPersistent.java}{FulfilledObligationCheck P}{8}{16}{FULp}

Infine se la scadenza è di tipo \emph{FACPLDate} allora si utilizzerà le FulfilledObligationTimeCheck.
Queste si comportano in modo analogo alle \emph{FulfilledObligationNumCheck}. La differenza che le contraddistingue è
l'utilizzo dei metodi \emph{before} e \emph{after} per i controlli sulla scadenza.
\javaFirstline{FulfilledObligationTimeCheck.java}{getObligationResult per le tipo T}{139}{156}{FULtGet}
Inoltre, al posto del metodo \emph{subExpiration} in \emph{GetObligationResult}, si richiama \emph{checkExpiration}.
La funzione è utilizzata in questo caso solo per richiamare il metodo di debug per mostrare
la differenza tra il tempo relativo alla determinata richiesta e la scadenza. In \ref{lst:FULtCheck} si
mostra come viene calcolato il divario tra la scadenza e il tempo.
\javaLastline{FulfilledObligationTimeCheck.java}{Time difference}{202}{205}{FULtCheck}
\section{Plugin Eclipse}
\label{sec:Plugin Eclipse}
La sintassi di FACPL è definita in Xtext e viene usato Xtend per la traduzione da \ac{FACPL} in Java.
Xtext~\cite{xtext} è un framework open-source per lo sviluppo di linguaggi di programmazione.
Xtend~\cite{xtend} è un linguaggio di programmazione basato sintatticamente e semanticamente su Java, ma
ha una sintassi più concisa e funzionalità addizionali come: inferenza sui tipi, overload degli operatori e estensione dei metodi.
\MyFigure{img/toolChain.png}{ToolChain di FACPL}{1}
\MyFigure{img/XtextGram.png}{Produzione per le Obligation}{1}
\MakeUppercase{è} compilato in codice Java, ha lo stesso sistema di tipi statico e integra tutte le librerie Java esistenti.
Il plugin utilizza Xtext e Xtend per rendere Eclipse un ambiente di sviluppo per \ac{FACPL} che offre funzionalità
come l'autocompletamento e l'highlighting del codice. In Figura~\ref{fig:img/toolChain.png} si mostra come da un file FACPL
possa generare il codice Java o XACML utilizzando le regole di traduzione scrtte in Xtend.\par
\MyF{img/XtextObl.png}{Obligation in Xtend}{1}{9.2cm}
Inizialmente è stata estesa la grammatica aggiungendo le produzioni per i tre tipi di ObligationCheck.
Successivamente sono state modificate le regole di traduzione in Xtend in modo da tradurre il codice scritto in FACPL in Java.
In Figura \ref{fig:img/XtextGram.png} si mostra la nuova produzione delle Obligation.
Dopo aver esteso la grammatica sono state aggiunte le regole di traduzione. In Figura \ref{fig:img/XtextObl.png} si può
vedere come verrà generata l'obbligazione nel file Java della politica in base al tipo di obbligazione usata nel file \ac{FACPL}.
Si presenta in Figura \ref{fig:img/XtextGen.png} il plugin e la funzionalità che offre la generazione del codice Java
partendo da un file \ac{FACPL}.
\MyFigure{img/XtextGen.png}{Plugin di FACPL}{1}
\section{Esempi in Java}
\label{sec:Esempi in Java}
In questa sezione si ripropongono gli esempi descritti in \ref{sec:Esempi} mostrando le classi Java relative
ai file FACPL.
\subsection{Servizio di streaming}
\label{sub:EsStreamingJava}
Le classi associate all'esempio di servizio streaming (\ref{sub:RW_Code_Sec})
sono StatusStreaming, le ContextRequest degli utenti Alice e Bob, PolicySet\_Streaming e MainFACPL.

La classe StatusStreaming contiene FacplStatus. Si fa notare che nel codice \ref{lst:statusStr},
lo status viene creato una sola volta e sarà unico per tutte le richieste. Se non è inizializzato
si creano gli attributi e si aggiungono al campo FacplStatus, mentre se è già inizializzato il metodo
\texttt{getStatus()} restituirà direttamente \texttt{status}.
\javaFirstline{checkStreamingExample/StatusStreaming.java}{Status}{9}{32}{statusStr}

Le richieste di ascolto e di login sono simili tra loro. Queste si differenziano in base agli attributi associati.
Nel codice \ref{lst:requestStr} si mostra la richiesta di ascolto di Alice come esempio di \emph{request}.
Per prima cosa si creano due attributi, uno per il nome dell'utente (Alice) e uno per il tipo di azione (listen).
Dopo la creazione, i due \emph{attributes} sono aggiunti alla \emph{request}. Infine si associa il contesto della
richiesta allo Status con il metodo \texttt{setStatus()} che prende come attributo FacplStatus usando \texttt{getStatus()}
di StatusStreaming. \par
\javaFirstline{checkStreamingExample/ContextRequest_ListenPremiumAlice.java}{Request}{10}{33}{requestStr}

Il codice \ref{lst:policyStr} è relativo alla Policy di ascolto dell'utente standard Bob.
Il target della policy è composto da due \emph{Expressions}. La prima è associata all'azione che deve essere una \texttt{listen},
la seconda invece si riferisce al nome dell'utente che deve essere \texttt{Bob}.\par
Si fa notare la creazione di una FacplDate (new FacplDate("00:15:00")) di quindici minuti per assegnare la scadenza all'obligation check.
Nell'obbligazione si valuta inoltre che l'utente sia ancora \texttt{Bob} e che il booleano
\texttt{streamingBob} (inizializzato a \texttt{true} solo dopo il login) sia \texttt{true}.
\javaFirstline{checkStreamingExample/PolicySet_Streaming.java}{Policy Set}{169}{200}{policyStr}
La regola da valutare associata al PolicySet è stata aggiunta con il metodo
\texttt{addPolicyElement(new Rule\_ListenBob())}.
Nella \texttt{Rule\_ListenBob}, mostrata nel listato \ref{lst:ruleStr}, si controlla che Bob abbia fatto il login e che
non debba ascoltare la pubblicità. Dopo il controllo si modifica il booleano \texttt{commercialsBob} a true in modo tale
che, dopo la scadenza della ObligationCheck, Bob non possa fare un'altra richiesta di ascolto.
\javaLastline{checkStreamingExample/PolicySet_Streaming.java}{\texttt{Rule\_ListenBob}}{202}{223}{ruleStr}

La classe MainFACPL nel codice \ref{lst:mainStr} incorpora PDP, PEPCheck e il PolicySet. Nel costruttore si inizializzano
il PDP con un Combining Algorithm (in questo caso \texttt{PermitUnlessDenyGreedy}) e l'insieme di policy, il PEP con l'algoritmo di
Enforcement (\texttt{DENY\_BIASED}), Combining Algorithm (\texttt{DenyOverridesCheck}) e il PDP.
\javaFirstline{checkStreamingExample/MainFACPL.java}{MainFACPL}{20}{35}{mainStr}
Nel metodo \texttt{main} si aggiungono le richieste nella lista \texttt{requests}, queste poi verrano
valutate nel ciclo proposto di seguito in \ref{lst:mainForStr}.
\javaLastline{checkStreamingExample/MainFACPL.java}{Valutazione}{60}{72}{mainForStr}

\subsection{Accessi in lettura e scrittura di file}
\label{sub:EsAccesJava}
In questa sezione si mostrano le parti di codice java più significative associate all'esempio descritto in \ref{sub:RW_Code_Sec},
inoltre si presentano i miglioramenti a livello prestazionale attraverso l'uso delle \emph{ObligationCheck}
nel processo valutativo.\par
\vspace{2mm}
I codici per le richieste e per la costruzione dello status in questo caso non saranno trattati
in quanto hanno la stessa struttura dei file precedenti e cambiano soltanto il nome degli attributi e il loro tipo.\par
Si presenta invece la policy sulle letture che utilizza l'obbligazione di tipo permanente. A differenza della
delle regole sull'ascolto di Bob mostrato nel codice \ref{lst:policyStr}, l'\emph{ObligationCheck} della lettura in
\ref{lst:policyFileCheck} non ha nessuna scadenza, ma solo due \emph{Expressions} in cui si indica il controllo sull'
insieme dei file in cui è possibile fare la lettura e l'azione concessa(\texttt{read}). Con Questo tipo di struttura
l'obbligazione non raggiunge mai la scadenza fintantoché gli attributi delle richieste passano la verifica delle due espressioni.
\javaFirstline{checkWriteExample/PolicySet_ReadWriteCheck.java}{Policy di lettura}{80}{107}{policyFileCheck}

Si mostrano adesso le differenze tra i tempi computazionali relativi a un insieme di policy che utilizza le \emph{ObligationCheck}
e uno che non le utilizza.\par
Nel codice \ref{lst:policyFileNoCheck} si presenta la policy duale di \ref{lst:policyFileCheck},
in questo caso le regole non fanno uso del nuovo tipo di obbligazione.
\javaLastline{checkWriteExample/PolicySet_ReadWrite.java}{Policy di lettura senza ObligationCheck}{83}{105}{policyFileNoCheck}

Nella classe MainFACPL si utilizzano i due insiemi di regole per controllare due liste di richieste differenti.
In \ref{lst:mainBench} si mostra il caso in cui si utilizza il PolicySet senza ObligationCheck per la prima lista.
Tutti i tempi di computazione vengono scritti nel file "testNoCheck.txt".
\javaFirstline{checkWriteExample/MainFACPL.java}{Main per i benchmarks}{53}{70}{mainBench}
Nella prima lista sono prensenti cento richieste di lettura, nella seconda il numero totale di richieste è lo stesso,
ma ogni otto letture c'è una scrittura e una domanda di interruzione delle modifiche.
Utilizzando il set di policy senza ObligationCheck il PDP controlla sempre ogni richiesta. Se invece si usa l'insieme di regole
con la nuova obbligazione, nella prima lista di richieste, dopo il primo controllo, il PDP non sarà più parte del processo
di valutazione, nella seconda lista invece il PDP controllerà il 30\% delle richieste.
\MyFigure{img/bench.png}{Computazione}{1}
Ogni lista di richieste è stata verificata cento volte per entrambi i tipi di Policy per un totale di diecimila valutazioni.

In figura \ref{fig:img/bench.png} le unità in verticale indicano i millisecondi impiegati per controllare cento richieste.
La linea blu e quella arancione indicano i tempi per la computazione utilizzando il PolicySet con le ObligationCheck
mentre le altre due indicano i tempi per l'insieme di regole che non utilizzano il nuovo paradigma.
Analizzando i tempi totali per valutare diecimila richieste, in figura \ref{fig:img/benchTot.png} si può vedere che
per la prima lista (solo richieste di lettura) il sistema impiega tre volte meno tempo utilizzando le ObligationCheck, mentre per la
seconda lista c'è un miglioramento del 35\% circa.

Come era ragionevole attendersi, le misurazioni confermano che, se le richieste di scrittura sono minori rispetto a quelle di lettura, la nuova gestione
alleggerisce il processo valutativo.
\MyFigure{img/benchTot.png}{Tempo totale}{1}
