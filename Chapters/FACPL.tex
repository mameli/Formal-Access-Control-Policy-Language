\myChapter{Formal Access Control Policy Language }
\label{chap:Formal Access Control Policy Language}
Il linguaggio \ac{FACPL} (fakpol) è stato creato come alternativa a \ac{XACML}. Come accennato nel in \ref{sec:PBAC},
l'organizzazione \ac{OASIS} ha ideato il linguaggio XACML per sviluppare sistemi basati sul modello \ac{PBAC}.\par
\ac{XML} è utilizzato da \ac{XACML} per definire le sue politiche. Questo linguaggio di markup è molto usato per lo scambio
di dati tra sistemi, ma rende le politiche difficili da comprendere. Infatti anche le regole più semplici sono prolisse
e questo rende la lettura problematica per un utente.\par
\ac{FACPL} nasce dalle idee di \ac{XACML} e ha l'obiettivo di semplificare la scrittura di politiche e di definire un framework
costruito sopra basi formali solide, in modo da permettere agli sviluppatori di specificare e verificare automaticamente
delle proprietà.

In questo capitolo si descrive il linguaggio delinea prima di tutto il processo di valutazione in \ref{sec:Il processo di valutazione}
poi mostrano le componenti della sistema in \ref{sub:Componenti del sistema} e il loro significato in \ref{sec:sem_fpl}. Infine si
mostrerà con un esempio l'utilizzo di \ac{FACPL} in \ref{sec:Esempio_FACPL}.
\section{Il processo di valutazione}
\label{sec:Il processo di valutazione}
In questa sezione si espone la sequenza di azioni che il sistema compie affinchè una richiesta in input sia valutata.
\MyFigure{img/evaluationProcess.png}{Processo di valutazione}{1}
In figura \ref{fig:img/evaluationProcess.png} si mostrano i passi che vengono eseguiti.
Le componenti chiave del processo sono tre:
\begin{itemize}
  \renewcommand\labelitemi{--}
  \item \ac{PR}
  \item \ac{PDP}
  \item \ac{PEP}
\end{itemize}
Si assume che un insieme di risorse sia in coppia con le Policy, e che queste definiscano le credenziali
necessarie per ottenere l'accesso. Il \ac{PR} contiene le Policy e le rende disponibili al \ac{PDP} (Step 1), il
quale decide se l'accesso viene garantito o meno. \par
Quando la richiesta è ricevuta dal \ac{PEP} (Step 2), le sue credenziali vengono codificate in una sequenza
di attributi(una coppia nome-valore) per poi utilizzare quest'ultima per creare la richiesta \ac{FACPL}(Step 3).\par
Il Context Handler aggiunge attributi dell'ambiente (come la temperatura esterna) alla richiesta e
poi invierà questa al \ac{PDP} (Step 4).\par
Il processo di autorizzazione del \ac{PDP} restituisce una \emph{risposta} verificando che gli attributi, che possono far
parte della richiesta o del contesto, siano conformi ai controlli delle policy (Step 5-8). La \emph{risposta} del \ac{PDP}
contiene una \emph{decisione} e una possibile \emph{obbligazione}(Step 9-10).\par
La \emph{decisione} può essere di quattro tipi:
\begin{itemize}
  \renewcommand\labelitemi{--}
  \item Permit
  \item Deny
  \item Not-applicable
  \item Indeterminate
\end{itemize}
I primi due tipi indicano rispettivamente richiesta accettata e richiesta non accettata. Se viene restituito Not-applicable
non ci sono policy su cui si poteva valutare la decisione, se ci sono altri errori la risposta è Indeterminate.
Le \ePolicy gestiscono automaticamente il risultato Indeterminate combinandolo con le altre decisioni
a seconda della strategia del \emph{Combining Algorithm} associato alla Policy stessa.\par
Le \eObligations sono azioni addizionali che devono essere eseguite dopo la restituzione di una decisione.
Solitamente corrispondono all'aggiornamento di un file, l'invio di un messaggio o l'esecuzione di un comando.
Il \ac{PEP} ha compito di controllare lo svolgimento delle \eObligations tramite l'\emph{obligation service} (Step 11-12).
Il processo di \emph{enforcement} eseguito dal \ac{PEP} determina l'\emph{enforced decision} in base al risultato delle obbligazioni.
Questa decisione può differire da quella del \ac{PDP} e corrisponde alla valutazione finale di tutto il processo.


\section{Componenti del sistema}
\label{sub:Componenti del sistema}
Nella tabella \ref{tab:facpl_syntax} si mostra la sintassi di \ac{FACPL}. Questa è data da una grammatica di tipo \ac{EBNF} dove il simbolo ? indica
elementi opzionali, * sequenze (anche vuote) e + sequenze non vuote.\par
Al livello più alto troviamo il termine  \ac{PAS} che comprende le specifiche del \ac{PEP} e del \ac{PDP}.\par
Il \ac{PEP} è definito con un \emph{enforcing algorithm} che sarà applicato per stabilire quali sono le decisioni che devono
passare al processo di \emph{enforcement}.\par
Il \ac{PDP} è definito come una sequenza di \emph{Policy}\textsuperscript{+} e da un algoritmo \emph{Alg} per combinare i
risultati delle valutazione delle policy.\par
\begin{table}[h]
\centering
\scriptsize
\caption{Sintassi di FACPL}
\hrule
$
\begin{array}{@{\,}r@{\ \ }r@{\ }r@{\ \ }l@{\ }}
{\textbf{Policy Authorisation Systems}} &
\mathit{PAS} & ::= & ( \,  \x{pep:} \, \mathit{EnfAlg}\ \ \x{pdp:}\, \mathit{PDP} \, )
\\[.2cm]
{\textbf{Enforcement algorithms}} &
\mathit{EnfAlg}
& ::= & \based \Sep \denyBiased \Sep \permitBiased
\\[.2cm]
{\textbf{Policy Decision Points}} &
\mathit{PDP} & ::= & \pdpPol{\algNT\ }{\x{policies:} \, \mathit{Policy}^{+}}
\\[.2cm]
{\textbf{Combining algorithms}} &
\algNT & ::= & \permitOver \Sep \denyOver \Sep \denyUnless \Sep \permitUnless \\
&& \mid &
\firstApp \Sep \onlyOneApp \Sep \weakCon \Sep \strongCon
\\[.2cm]
{\textbf{Policies}} &
\mathit{Policy} & ::= &
\ruleOpt{\mathit{Effect}\ \ \x{target:} \, Expr\ \ \x{obl:} \, \mathit{Obligation}^{*} \, } \\
&& \mid &
\{ \algNT\ \ \x{target:} \, Expr\ \
\x{policies:} \, \mathit{Policy}^{+} \ \ \x{obl:} \, \mathit{Obligation}^{*} \, \}
\\[.2cm]
{\textbf{Effects}} &
\mathit{Effect} & ::= & \permit \Sep \deny
\\[.2cm]
{\textbf{Obligations}} &
\mathit{Obligation} & ::= & [ \, \mathit{Effect} \ \ \mathit{ObType} \ \ \obl{Expr} \, ]
\\[.2cm]
{\textbf{Obligation Types}} &
\mathit{ObType} & ::= & M \Sep O
\\[.4cm]
\textbf{Expressions}&
\mathit{Expr} & ::= &
\mathit{Name} \Sep \mathit{Value}  \\
& & \mid &\x{and(\mathit{Expr}, \mathit{Expr})} \Sep \x{or(\mathit{Expr}, \mathit{Expr})} \Sep \x{not(\mathit{Expr})} \\
& & \mid &
 \x{equal(\mathit{Expr},\mathit{Expr})}  \Sep \x{in}(\mathit{Expr}, \mathit{Expr}) \\
& & \mid & \x{greater}\textrm{-}\x{than(\mathit{Expr},\mathit{Expr})} \Sep \x{add(\mathit{Expr} ,\mathit{Expr} )}\\
& & \mid & \x{subtract(\mathit{Expr} ,\mathit{Expr} )} \Sep \x{divide(\mathit{Expr} ,\mathit{Expr} )}\\
& & \mid & \x{multiply(\mathit{Expr} ,\mathit{Expr} )} \\[.2cm]
%
\textbf{Attribute Names} &
\mathit{Name} & ::= & \mathit{Identifier}/\mathit{Identifier} \\[.2cm]
%
\textbf{Literal Values} &
\mathit{Value} & ::= & \x{true} \mid \x{false} \mid \mathit{Double} \mid \mathit{String} \mid \mathit{Date}
\\[.4cm]
{\textbf{Requests}} &
\mathit{Request} & ::= & {\attribute{\mathit{Name}}{\mathit{Value}}}^{+}
\\[.1cm]
\end{array}
$
\hrule
\label{tab:facpl_syntax}
\end{table}
{\label{Sintassi di FACPL}}
Una \ePolicy può essere una \eRule semplice oppure un \ePolicySet cioè un insieme di rule o altri
policy set. In questo modo si possono creare regole singole, ma anche gerarchie di regole.\par
Un \ePolicySet è definito da un \etarget che indica l'insieme di richieste di accesso al quale la policy
viene applicata, da una lista di \eObligations cioè le azioni opzionali o obbligatoriche da eseguire, da una
sequenza di \ePolicy e da un algoritmo per la combinazione.\par
Una \eRule  è specificata da un \emph{effect} , che può essere permit o deny , da un \etarget
e da una lista di \eObligations(che può essere anche vuota). \par
Le \emph{Expressions} sono costituite da \emph{attribute names} e da valori letterali(per esempio booleani, stringhe, date).\par
Un \emph{attribute name} indica il valore di un attributo. Questo può essere contenuto in una richiesta o nel contesto. La
struttura di un attribute name è della forma \emph{Identifier}/\emph{Identifier}. Dove il primo elemento indica la categoria
e il secondo il nome dell'attributo. Per esempio Name / ID rappresenta il valore di un attributo ID di categoria Name.\par
Un \emph{combining algorithm} ha lo scopo di risolvere conflitti delle decisioni date dalle valutazioni delle policy.
Ad esempio permit-overrides assegna la precedenza alle decisioni con effetto permit rispetto a quelle di tipo deny.\par
Una \eObligation è definita da un effetto, da un tipo (M per obbligatorio e O per opzionale) e da un'azione e le relative
espressioni come argomento.\par

Una richiesta consiste in una sequenza di \emph{attribute} organizzati in categorie.\par
La risposta ad una richiesta \ac{FACPL} è scritta utilizzando la sintassi ausiliaria riportata in tabella \ref{tab:facpl_context_syntax}.
La valutazione in due passi descritta in \ref{sec:Il processo di valutazione} produce due tipi differenti di risposte:
\begin{itemize}
  \renewcommand\labelitemi{--}
  \item \emph{\ac{PDP} Response}
  \item \emph{Decisions}
\end{itemize}
La prima nel caso in cui la decisione sia permit o deny si associa a una sequenza (anche vuota) di fulfilled obligations.\par
Una \emph{Fulfilled Obligation} è una coppia formata da un tipo e da un'azione con i rispettivi argomenti risultato della
valutazione del \ac{PDP}.
\begin{table}[!t]
\small
\caption{Sintassi ausiliaria per le risposte}
$
\begin{array}{@{\ }r@{\ \ \ \ }r@{\ }r@{\ \ }l@{\ }}

&&&\\[-.2cm]
{\textbf{PDP \ Responses }} &
\mathit{PDPResponse} & ::= & \langle \,\mathit{Decision} \ \ \ \mathit{FObligation}^* \rangle
\\[.2cm]
{\textbf{Decisions}} &
\mathit{Decision} & ::= & \permit \Sep \deny \Sep \notApp \Sep \indet
\\[.2cm]
{\textbf{Fulfilled obligations}} &
\mathit{FObligation} & ::= &  [ \, \mathit{ObType} \ \ \obl{\mathit{Value}} \, ]\\[.1cm]

\end{array}
$
\label{tab:facpl_context_syntax}
\end{table}


\section{Semantica}
\label{sec:sem_fpl}
Si presenta adesso informalmente il processo di autorizzazione del \ac{PDP} e poi quello di enforcement del \ac{PEP}.\par
Quando il \ac{PDP} riceve una richiesta di accesso, valuta la richiesta in base alle \epolicy disponibili poi determina
il risultato unendo le decisioni restituite dalle policy con l'utilizzo dei \emph{combining algorithm}.\par
La valutazione di una policy inizia dalla verifica dell'applicabilità della richiesta, che è compiuta valutando
l'espressione definita dal \etarget. Ci sono due casi:
\begin{description}
  \item[\normalfont\bfseries{\MakeUppercase{L}a verifica dà un risultato positivo.}] \hfill \\
        Nel caso in cui ci siano \emph{rules}, l'effetto della regola viene restituito.
        Nel caso di \ePolicySet, il risultato è ottenuto dalla valutazione delle \epolicy contenute e dalla combinazione di queste,
        tramite l'algoritmo specificato dal \ac{PDP}. In entrambi i casi in seguito si procederà al \emph{fulfilment} delle
        \eobligation da parte del \ac{PEP}.
  \item[\normalfont\bfseries{\MakeUppercase{L}a verifica dà un risultato negativo.}] \hfill \\
        Nel caso in cui la valutazione sia determinata \efalse, viene restituito
        \texttt{not-app}. Nel caso di \emph{error} o di un valore non booleano, si restituisce \texttt{indet}.
\end{description}
Valutare le espressioni corrisponde ad applicare gli operatori, a determinare le occorrenze degli \emph{attribute names} e
a ricavarne il valore associato.\par
Se questo non è possibile, ad esempio l'attibuto è mancante e non può essere recuperato dal \emph{context handler}, si
restituisce il valore speciale \texttt{BOTTOM}. Questo valore è usato per implementare strategie diverse per gestire
la mancanza di un attributo. \texttt{BOTTOM} è trattato da \ac{FACPL} in modo simile a \texttt{false}, si gestisce
così gli attributi mancanti senza generare errori.\par
Gli operatori tengono conto degli argomenti e dei valori speciali come \texttt{BOTTOM} e \emph{error}.
Se gli argomenti sono \texttt{true} oppure \texttt{false} gli operatori sono applicati in modo regolare.
Se c'è un argomento \texttt{BOTTOM} e non ci sono \emph{error} si restituisce \texttt{BOTTOM}, \emph{error} altrimenti.
Gli operatori \eand e \eor trattano diversamente i valori speciali. Specificatamente, \eand restituisce \texttt{true} se entrambi
gli operandi sono \etrue, \texttt{false} se almeno uno è \efalse, \texttt{BOTTOM} se almeno uno è \texttt{BOTTOM} e nessun altro è
\texttt{false} o \emph{error}, \emph{error} negli altri casi. L'operatore \eor è il duale di \eand e ha priorità minore.
L'operatore unario \emph{not} cambia i valori di \texttt{true} e \texttt{false}, ma non quelli di \texttt{BOTTOM} e \emph{error}.\par
La valutazione di una \epolicy termina con il \emph{fulfilment} di tutte le \emph{Obligations}. Questo consiste nel valutare
tutti gli argomenti dell'azione corrispondente all'\emph{obligation}. All'occorrenza di un errore, la decisione della \epolicy
sarà modificata in \texttt{indet}.
Negli altri casi la decisione non cambierà e sarà quella del \ac{PDP} prima del \emph{fulfillment}.\par
Per valutare gli insiemi di \epolicy si devono applicare dei \emph{combining algorithm} specifici. Data una sequenza di
policy in input gli algoritmi stabiliscono una sequenza di valutazioni per le \epolicy date. Si propone in seguito l'algoritmo
\emph{permit-overrides} come esempio.
\begin{description}[labelindent=5pt,style=multiline,leftmargin=3cm]
\item[permit-overrides]
                        Se la valutazione di una \epolicy restituisce \emph{permit}, allora il risultato è \emph{permit}.
                        In altre parole, \emph{permit} ha la precedenza, indipendente dal risultato delle altre \emph{policy}.
                        Invece, se c'è almeno una \epolicy che restituisce \emph{deny} e tutte le altre restituiscono
                        \texttt{not-app} o \emph{deny}, allora il risultato è \emph{deny}. Se tutte le policy restituiscono
                        \texttt{not-app}, allora il risultato è \texttt{not-app}. In tutti gli altri casi, il risultato è \texttt{indet}.
\end{description}
Se il risultato della decisione è \emph{permit} o \emph{deny}, ogni algoritmo restituisce una sequenza di \emph{fulfilled obbligations}
conforme alla strategia $\delta$ di \emph{fulfilment} scelta. Ci sono due possibili strategie:
\begin{description}[labelindent=5pt,style=multiline,leftmargin=3cm]
  \item[All]La strategia \emph{all} richiede la valutazione di tutte le \epolicy appartenenti alla sequenza in input e
            restituisce le \emph{fulfilled obligation} relative a tutte le decisioni.
  \item[Greedy]La strategia \emph{greedy} stabilisce che, se ad un certo punto, la valutazione della sequenza di \epolicy
               in input non può più cambiare, si può non esaminare le altre policy e terminare l'esecuzione. In questo modo
               si migliora le prestazioni della valutazione in quanto non si sprecano risorse computazionali
               per valutazioni di policy che non avrebbero alcun impatto sul risultato finale.
\end{description}\par
L'ultimo passo consiste nell'inviare la risposta del \ac{PDP} al \ac{PEP} per l'\emph{enforcement}. A questo scopo, il \ac{PEP} verifica
che tutti gli eventuali obblighi imposti dal \ac{PDP} al richiedente siano soddisfatti e decide, in base all'algoritmo di
enforcement scelto, il comportamento per le decisioni di tipo \texttt{not-app} e \texttt{indet}. Gli algoritmi sono:
\begin{description}[labelindent=5pt,style=multiline,leftmargin=4cm]
  \item[base]Il \ac{PEP} mantiene tutte le decisioni, ma se c'è un errore nella verifica degli obblighi il risultato è \texttt{indet}.
  \item[deny-biased]Il \ac{PEP} concede l'accesso solo nel caso in cui questa sia la decisione del \ac{PDP} e che gli eventuali obblighi imposti dal \ac{PDP} siano stati soddisfatti. In tutti gli altri casi, il \ac{PEP} nega l’accesso.
  \item[permit-biased]
  Questo algoritmo è il duale di deny-biased.
\end{description}
Questi algoritmi evidenziano il fatto che le \emph{obligations} non solo influenzano il processo di autorizzazione, ma anche l'enforcement.
Si nota infine che gli errori causati da obbligazioni opzionali sono ignorati.


\section{Esempio FACPL}
\label{sec:Esempio_FACPL}
In questo semplice esempio di politica in \ac{FACPL}, due utenti possono interagire con il sistema attraverso delle richieste.
John può scrivere sulla risorsa "file.txt", ma non vi è specificata nessuna regola per la lettura. Invece Tom può
leggere il "file.txt", ma non può scriverci.\par
\lstinputlisting[language = FACPL, caption = {Esempio di politica in FACPL}\label{lst:facpl_es_Pol}]{./code/cap3/esempioFacpl}
Nel PAS si indicano gli algoritmi usati e le richieste da includere.
In questo esempio entrambi gli utenti richiedono sia l'azione di "WRITE" che l'azione di "READ".
Qui di seguito si presenta la struttura delle \emph{request}.
\lstinputlisting[language = FACPL, caption = {Esempio di richieste in FACPL}\label{lst:facpl_es_Req}]{./code/cap3/esempioRichiesteFacpl}
L'output dopo l'esecuzione sarà il seguente:
\begin{verbatim}
  Request: Request1

    Authorization Decision: PERMIT
    Obligations: PERMIT M log_permit([John])

  Request: Request2

    Authorization Decision: NOT_APPLICABLE
    Obligations:

  Request: Request3

    Authorization Decision: PERMIT
    Obligations: PERMIT M log_permit([Tom])

  Request: Request4

  Authorization Decision: DENY
  Obligations: DENY M log_deny([Tom])
\end{verbatim}
La prima richiesta di scrittura da parte di John viene chiaramente accettata, la sua richiesta di lettura però ha
come risultato \emph{NOT APPLICABLE} in quanto non ci sono regole che possono essere usate per dare una valida risposta.
La richiesta di lettuta di Tom invece viene accettata, mentre la sua ultima richiesta di scrittura riceve un deny
perché è bloccata alla \emph{rule} writeRuleT.\par
Da questo esempio si può vedere la semplicità con cui si possono scrivere le policy e le richieste.
Le stesse politiche scritte in \ac{XML} oltre a risultare prolisse a confronto, sono difficili da comprendere o analizzare.\par
Infine si fa notare che in questa versione, le richieste in ingresso sono completamente indipendenti tra loro e
la parte dello Usage Control descritta in \ref{sec:Usage_Control}, in cui si enfatizza l'uso di controlli continuativi e
la gestione di contesti mutabili, non è ancora implementata. Nel capitolo successivo si mostra la nuova struttura per
per uno sviluppo basato proprio sullo Usage Control.
