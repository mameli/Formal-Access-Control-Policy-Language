\myChapter{Estensione della libreria FACPL}
\label{chap:Estensione della libreria FACPL}
FACPL è basato su Java. Per implementare le nuove funzioni sono state aggiunte varie classi e sono state modificate
alcune parti per adattare il comportamento della libreria ai nuovi componenti e alle nuove strutture.\par
Nelle sezioni successive si descriveranno gli aspetti più rilevanti del codice. L'intera libreria si può trovare su
GitHub all'indirizzo
\center \url{https://github.com/andreamargheri/FACPL}
\\\par
Lo sviluppo dell'estensione si può dividere in due passi principali.\par
La prima parte consiste nel creare una classe PEP che può gestire le richieste e adattare il processo di valutazione.
La seconda parte invece comprende l'implementazione di una gerarchia di \emph{Obbligations} che possono essere
sfruttate dal PEP per la nuova gestione delle richieste.
\section{La classe PEPCheck}
\label{sec:PEPCheck}

\section{Le Obbligations Check}
\label{sec:Le Obbligations Check}
