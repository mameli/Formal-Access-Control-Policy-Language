\documentclass{article}

\usepackage[italian]{babel}
\usepackage[utf8]{inputenc}
\usepackage{hyperref}
\usepackage[normalem]{ulem}
\usepackage[font=large, tablename= ]{caption}
\useunder{\uline}{\ul}{}
\renewcommand{\baselinestretch}{1.3}
\setlength{\textwidth}{16 cm}
\setlength{\oddsidemargin}{0 cm}
\setlength{\topmargin}{-1.5 cm}
\setlength{\textheight}{30 cm}
\begin{document}

\begin{table}[]
\centering
\caption*{Progetto e implementazione in FACPL di un monitor a runtime per il supporto al controllo continuativo degli accessi}
\begin{tabular}{lll}
\textbf{Candidato:}   & Filippo Mameli & \href{mailto:filippo.mameli@stud.unifi.it}{\texttt{filippo.mameli@stud.unifi.it}}  \\
\textbf{Relatore:}    & Rosario Pugliese  & \href{mailto:rosario.pugliese@unifi.it}{\texttt{rosario.pugliese@unifi.it}}                           \\
\textbf{Correlatore:} & Andrea Margheri   & \href{mailto:andrea.margheri@unifi.it}{\texttt{andrea.margheri@unifi.it}}
\end{tabular}
\end{table}
\subsection*{Riassunto}
I sistemi informatici, fin dalla loro nascita, sono stati utilizzati per la gestione di dati.
Il tipo di informazioni gestite ha reso necessario la creazione di un sistema che le proteggesse, infatti i
dati più sensibili, se diffusi senza una valida autorizzazione, possono arrecare ad esempio danni economici
ad una società o nuocere anche gli utenti nel privato.
Per far fronte al problema sono stati sviluppati dei modelli per il controllo degli accessi. Tuttavia la quantità di dati
e la complessità dei sistemi moderni ha mostrato i limiti delle tecnologie concepite e questo fatto ha portato inevitabilmente
allo sviluppo di un nuovo approccio: lo UCON.

I ricercatori dell'università di San Antonio, Jaehong Park e Ravi Sandhu, hanno descritto un modello
di UCON che consente di ottenere decisioni durante l'accesso e di basare le valutazioni di una richiesta
sul comportamento passato.

In questa tesi si descrive un'estensione basata su UCON per FACPL.
Il problema principale da risolvere è la staticità del processo di valutazione del linguaggio.
Tutte le richieste di accesso ricevono una risposta attraverso un sistema di verifica sempre uguale e i controlli
risultano ridondati e in alcuni casi inutili.
Nella prima parte dello sviluppo è stata modificata la sintassi del linguaggio per adeguarla ad una nuova
gestione delle richieste, in seguito
è stata sviluppata un'estensione della libreria su cui si basa FACPL per modificare
il processo valutativo al seguito di richieste di accesso continuative.

\end{document}
